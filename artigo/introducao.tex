\chapter{Introdução}
\label{introducao}

Sistemas para análise de imagens digitais muitas vezes requerem, como entrada,
quadros de resolução superior ao que é possível ser capturado pelo sensor, como
por exemplo para reconhecer pessoas a distância \cite{parkparkkang2003}.
Geralmente, quando o resultado da captura de imagens não tem a quantidade de
informação necessária, a etapa seguinte de processamento não gera resultados
satisfatórios. A solução mais direta para este problema é trocar os sensores
utilizados na captura por modelos mais avançados capazes de capturar, com maior
definição, detalhes requeridos, como por exemplo através de câmeras PTZ
(\textit{pan}, \textit{tilt} e \textit{zoom}). Porém isto pode implicar em um
aumento significativo do custo do sistema ou pode ser impossível caso a
tecnologia não esteja disponível no mercado.

Outra possibilidade é a aplicação da técnica de super-resolução, a qual
consiste em gerar uma imagem com a resolução desejada a partir de um conjunto
de quadros adquiridos pelo sensor \cite{howtobreak}. Uma sequência de quadros
de um vídeo de uma única câmera ou um conjunto de imagens capturadas por
diferentes câmeras simultaneamente representam diferentes vistas de uma mesma
cena. As diferenças nas amostragens dos pixels de uma imagem para a outra
resultam na captura de diferentes regiões e tornam a aplicação de tais técnicas
possíveis.

Na ocorrência de deslocamento entre diferentes quadros, mesmo que de baixa
intensidade, \textit{i.e.} inferior a um pixel, a quantidade de informação é
aumentada devido aos diferentes detalhes obtidos \cite{parkparkkang2003}. Após
este deslocamento ser computado, os pixels são alinhados e projetados em uma
nova matriz de maior densidade representando a imagem, na qual cada posição é
equivalente a um pixel \cite{gonzalez}, de modo a reconstruí-la. Idealmente,
haveria infinitos quadros e cada um registraria um deslocamento diferente. Na
prática, as diferenças de quadros muito distantes temporalmente tornam a
confiabilidade da estimação do movimento muito baixa, limitando o número de
imagem as quais podem ser utilizadas. Este processo está ilustrado na figura
\ref{img_rafael}, com um exemplo aplicado a um vídeo de uma pessoa passeando em
um gramado, e na figura \ref{img_alignment}, de uma forma teórica.

Em aplicações práticas, a qualidade da reconstrução depende de quão bem os
quadros de entrada foram alinhados \cite{howtobreak}. A figura \ref{img_wesley}
ilustra os resultados da super-resolução na reconstrução de uma mesma cena, com
duas abordagens para alinhamento: preciso (Figura
\ref{img_wesley_bom}) e impreciso (Figura \ref{img_wesley_ruim}).
Como solução, métodos de fluxo ótico \cite{lukaskanade, hornschunck} têm sido
empregados nesta fase de alinhamento \cite{bakerkanade99}.


A super-resolução não está limitada apenas a duas dimensões, podendo também ser
aplicada em imagens de profundidade \cite{rosenbush} (Figura \ref{img_align3d}.
Modelos 3D têm sido gerado a partir de câmeras Tempo de Voo (ToF, \textit{Time
of Flight}) de baixo custo \cite{schuon2009, gevrekci, cui2010} e de sensores
baseados em luz estruturada \cite{cui2011} através da super-resolução.

Este trabalho apresenta uma visão geral sobre as etapas (aquisição, alinhamento
e reconstrução) do processo de super-resolução tanto em imagens coloridas (RGB)
quanto em imagens de profundidade, com foco em sistemas biométricos. Neste
trabalho trata-se a primeira etapa como resolvida, sendo consideradas
sequências de vídeo de um único sensor.  Há uma ênfase maior em sistemas de
reconhecimento facial, pois em geral requerem pequenas distâncias e a
cooperação do indivíduo. Porém, espera-se que a aplicação da super-resolução
possa permitir o reconhecimento a maiores distâncias e com menor cooperação.

Foram implementados algoritmos de super-resolução para imagens de profundidade
e coloridas. E os resultados preliminares obtidos são apresentados no capítulo
\ref{resultados}. São também discutidas as dificuldades encontradas no
desenvolvimento de métodos para tratar estas imagens e suas possíveis soluções.
