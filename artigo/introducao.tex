\chapter{Introdução}
\label{ch:introducao}

Ao longo da última década, a realidade da comunicação em rede entre computadores pessoais e servidores evoluiu
para uma grande parcela da população, deixando de servir apenas para compartilhamento de websites.
Tornou-se um completo ambiente utilizado por cidadãos para armazenamento de arquivos pessoais na núvem;
por bancos para efetuar transações monetárias; e por governos para trocas de informações preciosas.
Assim como é possível se conectar de uma máquina pessoal a um servidor \textit{web}, é possível fazer o contrário.
Qualquer máquina conectada a uma rede está teoricamente sujeita ao acesso de fora.
Roteadores e sistemas operacionais atuais estão equipados com ferramentas básicas para previnir acesso não autorizado
ao computador \cite{hongjuan10}. Ainda assim, há pessoas dedicadas a burlar tais proteções. Alguns desses invasores conseguem copiar
arquivos, instalar \textit{malware} e até ganhar controle sobre o sistema sem serem notados pelo dono.
\par A fim de cobrir esse ponto cego, foram criados sistemas de detecção de intrusão, ou IDS
(\textit{intrusion detection systems}). Seu objetivo é reconhecer comportamentos incomuns dentro de uma rede.
No meio acadêmico, muitos artigos foram publicados explorando algoritmos de \emph{machine learning} nesta aplicação.
\par Existem diversos tipos de ataques e diferentes protocolos de comunicação, mas a tarefa de se detectar uma
intrusão pode ser simplificada em um problema de classificação binária do comportamento do sistema, onde os dois
possíveis resultados seriam "normal" ou "intrusivo". Existem duas "escolas de pensamento" bem definidas no meio
acadêmico acerca do assunto:
\begin{enumerate}
    \item \textit{Signature-based} (baseado em assinatura) -- Esse é o método padrão e atualmente o único utilizado
    comercialmente. Ele possui baixa adaptabilidade a novas ameaças.
    \item \textit{Anomaly-based} (baseado em anomalia) -- Ainda uma área de estudo, possui alta taxa de
    Falso Positivo, mas boa adaptabilidade e será o foco deste trabalho.
\end{enumerate}
\par Nesta monografia, farei um estudo sobre os Sistemas de Detecção de Intrusão em Rede baseados em Anomalia, aos quais
 vou me referir pela sigla em inglês A-NIDS \footnote{Anomaly-based Network Intrusion Detection Systems} por questão de
 simplicidade. Abordarei ao longo do trabalho o conceito geral e esquematização de sistemas de detecção; os algoritmos
 de aprendizado de máquina que são comumente utilizados na área; os tipos de dados e características que podem ser
 adquiridos da rede; e por fim, efetuarei três testes para ver na prática o funcionamento dos algoritmos \emph{Support Vector
 Machine} (SVM) e \emph{Decision Tree} (DT). O primeiro teste consistirá de uma análise sobre os efeitos da variação do
 tamanho de uma base de dados ao se treinar o sistema. O segundo visa colocar à prova a teoria de que o uso de aprendizado
 de máquina possa ter boa adaptabilidade a novos tipos de ataque. Já o último teste, será uma avaliação dos efeitos da
 redução da dimensionalidade da base de dados.