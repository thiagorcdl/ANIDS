\chapter{Introdução}
\label{chap:introducao}

A Internet representa, atualmente, não apenas mais uma ferramenta de uso diário, mas um completo ambiente utilizado por
cidadãos para armazenamento de arquivos pessoais, por bancos para efetuar transações bancárias e por governos para
trocas de informações preciosas.
Assim como é possível se conectar de uma máquina pessoal a um servidor \textit{web}, é possível fazer o contrário.
Qualquer máquina conectada a uma rede está teoricamente sujeita ao acesso de fora.
Roteadores e sistemas operacionais atuais estão equipados com ferramentas básicas para previnir acesso não autorizado
ao computador. Ainda assim, há pessoas dedicadas a burlar tais proteções. Alguns desses invasores conseguem copiar
arquivos, instalar \textit{malware} e até ganhar controle sobre o sistema sem serem notados pelo dono.
\par A fim de cobrir esse ponto cego, foram criados sistemas de detecção de intrusão, ou IDS
(\textit{intrusiondetection systems}). Seu objetivo é reconhecer comportamentos incomuns dentro de uma rede.
No meio acadêmico, muitos artigos foram publicados explorando algoritmos de Machine Learning nesta aplicação.
\par Existem diversos tipos de ataques e diferentes protocolos de comunicação, mas o problema de se detectar uma
intrusão pode ser simplificado em um problema de classificação binária do comportoamento do sistema, onde os dois
possíveis resultados seriam "normal" ou "intrusivo". Existem duas "escolas de pensamento" bem definidas no meio
acadêmico acerca do assunto:
\begin{enumerate}
    \item \textit{Signature-based}, ou baseado em assinatura. Esse é o método padrão e atualmente o único utilizado
    comercialmente. Ele possui baixa adaptabilidade a novas ameaças.
    \item \textit{Anomaly-based}, ou baseado em anomalia. Ainda uma área de estudo, possui alta taxa de
    Falso-Positivo, mas boa adaptabilidade.
\end{enumerate}
Este artigo dará, posteriormente, foco aos conceitos e desenvolvimento de algoritmos de Machine Learning e Sistemas de
Detecção de Intrusão em Rede baseados em Anomalia (A-NIDS).
