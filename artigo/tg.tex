\documentclass[12pt,a4paper]{ufpr}

\usepackage[brazil]{babel}
\usepackage[utf8]{inputenc}
\usepackage{amssymb,amsmath}
\usepackage{multirow}
\usepackage{amssymb}
\usepackage{subcaption}
\usepackage{graphicx}
\usepackage{caption}
\usepackage{setspace}
\usepackage{diagbox}
\usepackage[table,xcdraw]{xcolor}

% n - numero de niveis de subsubsection numeradas
\setcounter{secnumdepth}{3}
% coloca ate o nivel n no sumario
\setcounter{tocdepth}{3}
\graphicspath{{img/}}

\title{Sobre Aprendizagem de Máquinas aplicada a Sistemas de Detecção de Intrusão em Rede baseados em Anomalias}
\author{Thiago Roscia Cerdeiro de Lima}
% ou Orientador
\advisortitle{Orientador}
\advisorname{Prof. Dr. Luis Allan Kunzle}
% departamento, instituicao
\advisorplace{Departamento de Informática, UFPR}
\city{Curitiba}
\year{2015}

% nao insira o nome do orientador, ja eh feito automaticamente
\banca{}{}{}{}{}{}{}{}

\defesa{04 de outubro de 2000}

\notaindicativa{Monografia apresentada para obtenção do Grau de
Bacharel em Ciência da Computação pela Universidade Federal do
Paraná.}

\begin{document}

\makecapa
\makerosto         % cria folha de rosto para versao final da UFPR %
%\maketermo      % cria folha com o termo de aprovacao da dissertacao%

%\singlespacing           % espacamento 1 - capa UFPR%
%\onehalfspacing          % espacamento 1/2 %
\doublespacing            % espacamento 2 - UFPR %

\pagestyle{headings}
\pagenumbering{roman}

%\chapter*{Agradecimentos}
%\input{agradecimentos.tex}          % possiu somente o texto

\tableofcontents

%\listoffigures         % se houver mais do que 3 figuras
%\addcontentsline{toc}{chapter}{\MakeUppercase{Lista de Figuras}}
%\newpage

%\listoftables        % se houver mais do que 3 tabelas
%\addcontentsline{toc}{chapter}{\MakeUppercase{Lista de Tabelas}}
%\newpage

\chapter*{Resumo}
\addcontentsline{toc}{chapter}{\MakeUppercase{Resumo}}
%This paper presents a summary, which aims to cover the main ideas available on the
%literature of the past few years regarding network intrusion detection. The goal is to
%explain the general architecture of Network Intrusion Detection Systems (NIDS) and how
%different machine learning algorithms are applied. The focus of this review will be the
%anomaly-based systems, which present a vast field of study.
%
%
%\noindent \textbf{key-words}: nids, network security, intrusion detection, machine learning, data mining, pattern recognition

Este trabalho de graduação tem como intuito apresentar um estudo acerca da usabilidade e viabilidade de
algoritmos de \textit{data mining} e \textit{machine learning} no contexto de sistemas de detecção de
intrusão em rede. Sua utilização teria grande impacto na segurança tanto de servidores quanto de clientes,
pois promete evitar ataques ao sistema detectando anomalias na rede.
Nesta monografia são apresentadas técnicas de \textit{machine learning}; a arquitetura geral
de um sistema baseado em anomalia; o processo de extração e processamento de dados; e, por fim, demonstra
os impactos de uma base de dados representativa, de tipos de ataques desconhecidos e da seleção de atributos.
Os estudos sugerem que a imprevisibilidade de novos tipos de ataques ainda é um empecilho, mesmo para
sistemas com capacidade de aprender ao ser alimentado com novos dados.

\noindent \textbf{Palavras-chave}: NIDS, network security, intrusion detection, machine learning, data mining,
pattern recognition, segurança, aprendizado de máquina, mineração de dados
           % somente o texto
\newpage


\pagenumbering{arabic}

\chapter{Introdução}
\label{chap:introducao}

A Internet representa, atualmente, não apenas mais uma ferramenta de uso diário, mas um completo ambiente utilizado por
cidadãos para armazenamento de arquivos pessoais, por bancos para efetuar transações bancárias e por governos para
trocas de informações preciosas.
Assim como é possível se conectar de uma máquina pessoal a um servidor \textit{web}, é possível fazer o contrário.
Qualquer máquina conectada a uma rede está teoricamente sujeita ao acesso de fora.
Roteadores e sistemas operacionais atuais estão equipados com ferramentas básicas para previnir acesso não autorizado
ao computador. Ainda assim, há pessoas dedicadas a burlar tais proteções. Alguns desses invasores conseguem copiar
arquivos, instalar \textit{malware} e até ganhar controle sobre o sistema sem serem notados pelo dono.
\par A fim de cobrir esse ponto cego, foram criados sistemas de detecção de intrusão, ou IDS
(\textit{intrusiondetection systems}). Seu objetivo é reconhecer comportamentos incomuns dentro de uma rede.
No meio acadêmico, muitos artigos foram publicados explorando algoritmos de Machine Learning nesta aplicação.
\par Existem diversos tipos de ataques e diferentes protocolos de comunicação, mas o problema de se detectar uma
intrusão pode ser simplificado em um problema de classificação binária do comportoamento do sistema, onde os dois
possíveis resultados seriam "normal" ou "intrusivo". Existem duas "escolas de pensamento" bem definidas no meio
acadêmico acerca do assunto:
\begin{enumerate}
    \item \textit{Signature-based}, ou baseado em assinatura. Esse é o método padrão e atualmente o único utilizado
    comercialmente. Ele possui baixa adaptabilidade a novas ameaças.
    \item \textit{Anomaly-based}, ou baseado em anomalia. Ainda uma área de estudo, possui alta taxa de
    Falso-Positivo, mas boa adaptabilidade.
\end{enumerate}
Este artigo dará, posteriormente, foco aos conceitos e desenvolvimento de algoritmos de Machine Learning e Sistemas de
Detecção de Intrusão em Rede baseados em Anomalia (A-NIDS).

\chapter{Sistemas baseadoes em Assinatura}
\label{ch:snids}
Apesar de não ser o foco desta monografia, vamos descrever brevemente os sistemas baseados em assinatura, a fim de
comparação com os métodos nos quais vamos nos aprofundar.
\par Sistemas baseados em assinatura têm sido os mais bem sucedidos para detecção de intrusão até hoje.
A ideia é comparar o tráfego da rede com padrões de comportamento conhecidos durante certos ataques.
Um grande ponto em aberto é a falta de eficácia em se detectar ataques novos \cite{holm14}.
\par Os primeiros sistemas utilizavam apenas o método de reconhecimento de padrão. Essa técnica possui um banco de dados
 preenchido com assinaturas de cada ameaça conhecida. Um evento malicioso é detectado se o atual estado da rede se iguala
 a uma assinatura. Como ele tenta comparar todas as assinaturas e a quantidade de dados transmitidos simultaneamente
cresce a cada ano, o custo computacional tronou-se muito alto.
\par Uma segunda técnica é o método de \textit{implication rules}, ou regras de inferência. Ela fornece um conjunto de
regras que descrevem eventos conhecidos que podem inferir o acontecimento de uma intrusão. De qualquer modo, IDS de
assinatura geralmente necessita de um humano capacitado, que cria um novo conjunto de regras toda vez que um novo
tipo de ataque surge, para adquirir modelos de tráfego.
\par Daí surge a motivação para se utilizar técnicas de \textit{data mining} em sistemas de assinatura \cite{han02}. Elas proveem um
 modo de aumentar a automação no momento de construção e de ajuste do modelo. Eles podem adaptar os modelos à medida que
 acessam tráfego de rede contendo novas ameaças ou detectando diferentes versões de um ataque conhecido. Ainda assim,
 não há perspectivas para se identificar comportamentos maliciosos completamente desconhecidos.
\chapter{Anomaly-based IDS}
\label{anids}
 A técnica de detecção de anomalia analisa o atual comportamento da rede para checar se corresponde ou não a um
 comportamento normal. O maior benefício desse método é a habilidade de identificar novos ataques com sucesso.
 A desvantagem é a alta taxa de Falso Positivo. Ao invés de requerer um novo modelo a ser adaptado, sistemas baseados
 em anomalia necessitam de dados de tráfego de rede sem livres de ataques. \textit{Data mining} também tem sido
 utilizado em sistemas baseados em anomalia, junto a outros métodos estatísticos e de \textit{Machine Learning}
 (aprendizado de máquina).
 \par As diferentes técnicas podem ser agrupadas em três categorias, de acordo com o processamento envolvido:
 estatístico, knowledge-based (baseado em conhecimento) ou algoritmos de aprendizado de máquina.

\section{Técinas Estatísticas}
Um modelo de probabilidade de um determinado comportamento é criado a partir da atividade capturada do tráfego de rede,
 tal como taxa de tráfego, número de pacotes e quantidade de endereços de IP distintos.
 \par Ele não requer conehcimento prévio pois está capacitado a aprender o comportamento normal a partir de observações
 sem ataques. Outro benefício das técnicas estatísticas é a possibilidade de identificar, com precisão, atividades
 maliciosas que ocorrem ao longo de períodos de tempo mais extensos. Por outro lado, nem todos os possíveis
 comportamentos conseguem ser modelados e o equilíbrio entre taxas de Falso Positivo e Falso Negativo dependem
 fortemente na configuração correta dos parâmetros.

 \section{Técnicas Baseadas em Conhecimento}
 Também chamados de expert systems, sistemas baseados em conhecimento são implementados criando-se um conjunto de
 regras de classificação para categorizar os dados. O modelo é geralmente criado por um humano experiente no campo
 da aplicação. Esse tipo de sistema não aponta novas atividades inofensivas como sendo maliciosas, garantindo, assim,
 um número reduzido de Falso Positivo. Por essa razão, o conjunto de regras precisa ser específico o suficiente e,
 apesar de ser possível atingir um certo nível de automatização usando uma máquina de estados finitos, requer
 grande conhecimento sobre o comportamento da rede e tempo significativo para ser desenvolvido.

 \section{Técnicas de Machine Learning}
\chapter{Dados}
\label{dados}

\section{Pré-processamento}
Como anteriormente apontado, sistemas baseados em anomalia estão sujeitos a uma alta taxa de Falso Positivo. Apenas
1\% sequer de Falso Positivo resulta em um número absurdo de falsos alertas. Deve-se ter em mente que servidores lidam
com centenas de conexões simultâneas e inúmeros pacotes a cada centésimo de segundo.
\par O pré-processamento de dados é, então, exigido para atingir uma melhor performance na detecção de intrusão.
O pré-processamento converte tráfego de rede em uma série de ocorrências, onde cada uma é representada por um vetor
de características. As informações seguintes acerca dos dados e características foram baseadas nos estudos presentes
em \cite{davis11}.

\subsection{Fluxo}
Os passos mais relevantes para o modelamento de dados de um NIDS são os seguintes:
\begin{enumerate}
    \item Criação da base -- identifica dados categorizados (normal ou anômalo) representativos para treino e teste.
    Categorização de tráfego de rede pode ser uma tarefa difícil e longa, que geralmente envolve um especialista.
    \item Construção de características -- cria características com maior discriminabilidade. Tais características
    podem ser construídas por um humano ou por algoritmos de aprendizado de máquina.
    \item Redução -- também chamado de "seleção de caterísticas", diminui a dimensionalidade da base de dados
    descartando características irrelevantes ou redundantes. Usado para atenuar a "Maldição da Dimensionalidade"
    \footnote{\textit{Curse of Dimensionality}: Ao se usar muitas características, supostamente se obtém baixa precisão.}
\chapter{Análise de Trabalhos Relacionados}
\label{ch:analise}

NIDS baseado em anomalia tem se mostrado uma área de estudo promissora, mas ainda possui grande margem para pesquisa e melhorias.
A fim de analisar o estado da arte, considerei o estudo feito em \cite{tsai09}, que traz dados de 55 artigos publicados
entre os anos 2000 e 2007. A partir dessa análise, pude extrair algumas inferências sobre as diferentes técnicas utilizadas.
\begin{itemize}
    \item  26 se baseiam no uso de classificadores unitários, apesar de que o uso de
    classificadores híbridos tem aumentado durante os últimos anos.
    Classificadores compostos são usados por poucos, totalizando apenas 6.

    \item Os algoritmos unitários mais estudados para detecção de instrusão foram SVM (7) e K-NN (6).

    \item Quanto a classificadores híbridos, o método \textit{integrado} é o mais popular, consistindo de 10 dos 23.

    \item A utilização de seleção de atributos tem crescido com o tempo. Enquanto em 2003 apenas um quarto dos estudos analisados
por \cite{tsai09} utilizava tal técnica, em 2007 foi o caso de 9 dentre 11. Ou seja, apenas dois (\cite{peddabachigari07}, \cite{li07}) não utilizaram.

    \item Como há poucas bases de dados públicas para detecção de instrusão, a maioria dos estudos se baseia nas
    bases KDD Cup 99 (30) e DARPA 1998 (18).
\end{itemize}

\par Quanto aos dados, as técnicas pioneiras costumavam criar modelos considerando todo o corpo do pacote. Entretanto,
eles separavam as instâncias com base no tamanho do pacote e portas de destino, diminuindo o contexto de cada modelo.
Técnicas posteriores levam em consideração partes específicas do conteúdo do pacote. Aparentemente, anomalias mais
sutis podem ser detectadas a uma taxa menor de Falso Positivo quando usando um contexto mais específico.
\par Dois artigos recentes trouxeram implementações promissoras a respeito de sistemas de detecção de
anomalia. Um deles (\cite{lin12}) apresenta um algoritmo inteligente e o outro (\cite{papadonikolakis12}) mostra o
ganho de desempenho ao se programar o sistema diretamente em uma placa FPGA. Para os fins desta monografia, não irei
implementar o algoritmo em FPGA e, apesar de o foco não ser classificadores em cascata, os estudos relacionados merecem
notoriedade por consituirem o estado da arte no que se propõem a fazer.

\section{Processando grandes quantidades de dados}
A simulação apresentada em \cite{lin12} usou a base do KDD Cup 99. O algoritmo funciona pré-processando os dados de treino e teste e gerando uma solução
inicial aleatória. A solução é atualizada a cada iteração do algoritmo. Enquanto um limite não é satisfeito,
são utilizados Support Vector Machine (SVM) e Simulated Annealing (SA) para selecionar o melhor conjunto de
características. Por conseguinte, Árvore de Decisão e SA são usados para aumentar a precisão do teste e construir
regras de decisão. No final, a melhor precisão e as melhores características e regras de decisão são anunciadas.
\par Nos testes descritos no  artigo, após a seleção de atributos, foram utilizadas 23 características durante o processo de
classificação. O algoritmo conseguiu atingir uma precisão de 99,6\%, o que representa um grande ganho quando comparado
 a outros resultados até o momento.

%%%%%%%%%%%%%%%%%%%%%%%%%%%%%%%%%%%%%%%%%%%%%%%%%%%%%%%%%%%%%%%%%%%%%%%%%%%%

\section{Aumentado a velocidade de processamento}
Enquanto o algoritmo anterior pretende otimizar apenas a precisão, o estudo apresentado em \cite{papadonikolakis12}
foca em na aceleração do processo de classificação. Aqui também foi utilizado SVM para implementar o A-NIDS, mas dessa
vez diretamente em um FPGA, fazendo uso de seu potencial de processamento paralelo.
\par Foram propostas três arquiteturas diferentes.
\begin{enumerate}
    \item \textit{Heterogeneous Baseline Classifier} -- O objetivo é explorar os requerimentos de precisão dos atributos. Ele
    tenta maximizar paralelização enquanto mantém a relação entre blocos de processamneto digital e blocos lógicos.
    \item \textit{Fit Cascade Chain} -- Aproveita o potencial de aritmética customizada oferecido pelo FPGA para criar uma
    cascata com duas precisões aritméticas diferentes. Primeiro um classificador de baixo custo e alta vazão com baixa
    precisão é usado. Somente os dados que não puderem ser classificados são alimentados ao segundo classificador. Este
    possui alta precisão e, logo, maior custo, mas com menos requisições de vazão.
    \item \textit{Reconfigurable Cascade Chain} -- Foi projetado para problemas com grandes quantidades de dados, que podem
    exceder as limitações de hardware. Essa arquitetura propõe uma reconfiguração do FPGA depois do estágio de treino e
    expande o espaço de design possível. Isso permite a utilização de mais blocos durante as classificação.
\end{enumerate}
\par A base utilizada para testes foi a \textit{MINIST Dataset}. Comparando com testes prévios em GPU, o design
\textit{Heterogeneous Baseline Classifier} processou os dados 7 vezes mais rápido. Os melhores resultados foram
obtidos usando \textit{Reconfigurable Cascade Chain}, com um ganho de 20 vezes, isto é, levando apenas 5\% do tempo
que a GPU consumiu.
\chapter{Testes}
Três baterias de testes foram conduzidas ao longo da produção deste trabalho de graduação a fim de ver na prática o
impacto dos algoritmos de aprendizado de máquina quando aplicados em bases de dados com informações de rede.
Em um dos testes, a base utilizada foi a supracitada KDD Cup 99. Para os outros, utilizou-se uma série de dados
 gerados pelo mestrando da PUC-PR Eduardo, orientado pelo Pr. Dr. Altair. Em ambos os casos, os algoritmos foram
 executados através do Weka.

\section{Weka}
O \textit{Waikato Environment for Knowledge Analysis} (Weka) é uma suíte de aplicações para aprendizagemd e máquina e
mineiração de dados. O software foi desenvolvido em Java na \textit{University of Waikato} e lincenciado sob a
\textit{GNU General Public License}. Os destaques dessa ferramenta incluem a vasta quantidade de algoritmos
implementados, pré-processamento de dados, seleção de atributos e visualização de dados. O Weka funciona como uma
interface, permitindo que novos algoritmos em Java sejam produzidos e facilmente executados de forma gráfica e com
resultados interativos \cite{bouckaert10}.

\section{Avaliação de Performance}
Gráficos ROC (\textit{Receiver Operating Characterístics}) são comumente usados como um método  para visualizar
a performance de classificadores e analisar as taxas de acertos e alarmes falsos \cite{fawcett04}. A Um gráfico ROC
baseia-se em apenas duas classes: positivo ou negativo. Em termos de detecção de intrusão, uma conexão pode ser
considerada inofensiva (positivo) ou prejudicial (negativa). Para avaliar a precisão de um classificador,
duas outras categorias podem ser usadas: verdadeiro ou falso. Quando cruzadas, obtém-se quatro possíveis resultados,
como mostra a tabela \ref{table:tfpn}.

\begin{table}[h]
    \begin{tabular}{l|l|l|}
        \cline{2-3}
                                                               & \cellcolor[HTML]{EFEFEF}Verdadeiro (T)                                                           & \cellcolor[HTML]{EFEFEF}Falso (F)                                                     \\ \hline
        \multicolumn{1}{|l|}{\cellcolor[HTML]{EFEFEF}Positivo} & \begin{tabular}[c]{@{}l@{}}TP -- Conexão inofensiva é \\ classificada corretamente\end{tabular}  & \begin{tabular}[c]{@{}l@{}}FP -- Conexão anômala é \\ considerada normal\end{tabular} \\ \hline
        \multicolumn{1}{|l|}{\cellcolor[HTML]{EFEFEF}Negativo} & \begin{tabular}[c]{@{}l@{}}TN -- Conexão prejudicial é \\ classificada corretamente\end{tabular} & \begin{tabular}[c]{@{}l@{}}FN -- Conexão normal é \\ considerada anômala\end{tabular} \\ \hline
    \end{tabular}
    \caption{Categorias para análise de corretude}
    \label{table:tfpn}
\end{table}

Quando se sabe a verdadeira natureza dos casos de teste, é possível determinar a precisão, a taxa de Verdadeiro Positivo
 ($TP_{rate}$) e taca de Falso Positivo ($FP_{rate}$). É possível inferir que uma das maiores preocupações ao se construir
 um A-NIDS é a minimização a taxa de Falso Positivo. Teoricamente se 100\% das conexões forem consideradas anômalas
 e bloqueadas, teremos $FP_{rate}$ igual a 0. Por um lado, o sistema estaria bloqueando conexões inofensivas, por outro,
 estaria com certeza evitando conexões ameaçadores. O cálculo das taxas se dá da seguinte maneira.

 $$ Precisão = \frac{TP + TN}{P + N} $$
 $$ TP_{rate} = \frac{TP}{P} $$
 $$ FP_{rate} = \frac{FP}{N} $$

Essas taxas estão contidas no espaço entre $0$ e $1$, inclusivo, e são utilizadas para plotar o ROC. A $TP_{rate}$ é
usado no eixo $y$ e a $FP_{rate}$, no eixo $x$. Portanto, o gráfico ROC elicita a relação entre Verdadeiro Positivo e
Falso Positivo.
\par A \ref{fig:roc} mostra uma curva ROC simples com 5 classificadores fictícios: A, B, C, D e E. Cada classificador
retorna um único ponto no espaço ROC. Se o ponto estiver localizado na coordenada (0,0), como o classificador A na
\ref{fig:roc}, significa que nenhum caso foi considerado inofensivo. Classificadores que se encontram próximos ao eixo
$y$ podem ser considerados "conservadores", uma vez que necessitam de forte evidência para categorizar uma ocorrência
como positiva. Em contrapartida, o ponto B, posicionado em (1,1), representa sistemas que não acusaria nenhuma
conexão como anômala. Classificadores mais À direita do gráfico podem ser considerados "liberais" pois presumem como
positivo mais facilmente.
\par Um classificador perfeito resultaria no ponto (0,1), onde C se encontra. Isso significaria possuir $FP_{rate}$ de
0\% e $TP_{rate}$ de 100\%, ou seja, todas as conexões intrusivas serem detectadas e nenhuma conexão normal ser
bloqueada pelo sistema. Assim sendo, quanto maior o $TP_{rate}$ e menor o $FP_{rate}$, mais preciso é o algoritmo.
 Se um classificador está localizado abaixo da linha diagonal ($y = x$), isto significa que sua performance é pior do
 que a escolha aleatória de classes. O triângulo inferior direito é geralmente encontrado vazio pois em casos de
 classificação binária com alta taxa de erro, é possível simplesmente inverter os resultados. Portanto, E (0.8,0.2)
 seria tão preciso quanto D(0.2,0.8)


\section{Impacto de modelos representativos}
O primeiro teste realizado levou em consideração apenas o SVM padrão fornecido pelo libSVM para o Weka. A finalidade
era verificar o impacto da representabilidade da base de treino \cite{yaman11}, analisando a performance do algoritmo
quando treinado com bases de diferentes tamanhos. Para esse estudo, o KDD Cup 99 foi divido em 5 particionamentos,
cada um com tamanhos distintos de bases de treino e de testes, usando a opção de \textit{cross-validation} do Weka.
O comando executado incluiu um pré-processamento de heurística de encolhimento, peso $1$, \emph{seed} de $3$, erro de
$0.001$ e função núcleo \textit{Radial Basis}.

\begin{table}[h]
    \begin{tabular}{l|l|l|l|l|}
        \cline{2-5}
                                                        & \multicolumn{1}{c|}{\cellcolor[HTML]{EFEFEF}Tamanho Treino} & \multicolumn{1}{c|}{\cellcolor[HTML]{EFEFEF}Tamanho Teste} & \multicolumn{1}{c|}{\cellcolor[HTML]{EFEFEF}Tempo} & \multicolumn{1}{c|}{\cellcolor[HTML]{EFEFEF}Precisão} \\ \hline
        \multicolumn{1}{|l|}{\cellcolor[HTML]{EFEFEF}1} & 10\%                                                        & 90\%                                                       & 16h13m                                             & 92.22\%                                               \\ \hline
        \multicolumn{1}{|l|}{\cellcolor[HTML]{EFEFEF}2} & 20\%                                                        & 80\%                                                       & 24h02m                                             & 94.54\%                                               \\ \hline
        \multicolumn{1}{|l|}{\cellcolor[HTML]{EFEFEF}3} & 30\%                                                        & 70\%                                                       & 27h32m                                             & 95.65\%                                               \\ \hline
        \multicolumn{1}{|l|}{\cellcolor[HTML]{EFEFEF}4} & 40\%                                                        & 60\%                                                       & 32h11m                                             & 98.62\%                                               \\ \hline
        \multicolumn{1}{|l|}{\cellcolor[HTML]{EFEFEF}5} & 50\%                                                        & 50\%                                                       & 38h45m                                             & 99.64\%                                               \\ \hline
    \end{tabular}
    \caption{Comparação entre particionamentos de tamanhos diferentes}
    \label{table:partic}
\end{table}

\par A \ref{table:partic} mostra os as proporções de base de testes e treino em cada particionamento e seus respectivos
resultados. É possível notar uma clara correlação entre o tamanho da base de treino e a porcentagem de acertos. O
melhor resultado foi de $99.64$\% quando metade da base foi usada para gerar o modelo. Entretanto, em uma situação real
não dados disponíveis o suficiente para se representar metade das transmissões que ocorrem diariamente.


\section{Impacto da imprevisibilidade}
Uma abordagem recente direcionou os estudos desse trabalho para a análise da real capacidade de algoritmos de
aprendizado de máquinas em se adaptar a novas realidades e  detectar ameaças totalmente desconhecidas \cite{sommer10}.

\section{Impacto da seleção de atributos}

\chapter{Conclusão}
\label{conclusao}

Neste trabalho foram realizados estudos sobre o tema super-resolução de forma a
avaliar sua aplicabilidade em dois cenários. O primeiro em imagens RGB de
sequência de vídeo contendo uma pessoa, de forma a fornecer quadros de maior
resolução a sistemas biométricos. Segundo em imagens de profundidade com
objetos rígidos, a fim de verificar o potencial da aplicação futura com
biometria.

Através da codificação de dois algoritmos, um para cada modalidade,
mostrou-se que o método, mesmo baseado em um modelo simplificado de observação,
permite a obtenção de resultados melhores no aumento da resolução espacial de
imagens quando comparado a soluções tradicionais, como a interpolação bicúbica.

A confirmação da eficácia da super-resolução em sequências de vídeo contendo
faces indica o potencial de sua aplicação em sistemas biométricos. Podendo
tanto diminuir o custo, quanto permitir maior performance.

Em trabalhos futuros espera-se considerar outros tipos de movimento além da
translação; tratar oclusões; experimentar e comparar outros métodos de
estimação de movimento na etapa de alinhamento; avaliar os resultados obtidos
com diferentes métodos de reconstrução; incluir estimação de ruído e desfoque;
e quantificar a influência da super-resolução no aumento da performance de
sistemas biométricos, como por exemplo, de reconhecimento facial.

% Trocar para ficar no padrão brasileiro
\bibliographystyle{brazil}
%\bibliographystyle{plain}

\inputencoding{latin2}
\bibliography{bib}
\inputencoding{utf8}
% utilize macros (3 primeiras letras do mes em ingles, minusculas) no seu
% .bib para atribuir o nome do mes em portugues nas referencia,
% se o style for brazil, outros estilos tambem aceitam estas macros
% Ex:
%
% @InProceedings{teste,
%   author =       {Luciano}
%   year =         {2000}
%   month =        {}#sep;
% }
%
\addcontentsline{toc}{chapter}{\MakeUppercase{Bibliografia}}

\singlespacing

\end{document}
