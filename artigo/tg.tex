\documentclass[12pt,a4paper]{ufpr}

\usepackage[brazil]{babel}
\usepackage[utf8]{inputenc}
\usepackage{amssymb,amsmath}
\usepackage{multirow}
\usepackage{amssymb}
\usepackage{subcaption}
\usepackage{graphicx}
\usepackage{caption}
\usepackage{setspace}
\usepackage{diagbox}
\usepackage[table,xcdraw]{xcolor}
\usepackage{pgfplots}

% n - numero de niveis de subsubsection numeradas
\setcounter{secnumdepth}{3}
% coloca ate o nivel n no sumario
\setcounter{tocdepth}{3}
\graphicspath{{img/}}

\title{Machine Learning em Sistemas de Detecção de Intrusão em Rede baseados em Anomalias}
\author{Thiago Roscia Cerdeiro de Lima}
% ou Orientador
\advisortitle{Orientadores}
\advisorname{\\Prof. Dr. Luis Allan Kunzle \\Prof. Dr. Luiz Eduardo S. Oliveira}
% departamento, instituicao
\advisorplace{Departamento de Informática, UFPR}
\city{Curitiba}
\year{2015}

% nao insira o nome do orientador, ja eh feito automaticamente
\banca{}{}{}{}{}{}{}{}

\defesa{28 de junho de 2015}

\notaindicativa{Monografia apresentada para obtenção do Grau de
Bacharel em Ciência da Computação pela Universidade Federal do
Paraná.}

\begin{document}

\makecapa
\makerosto         % cria folha de rosto para versao final da UFPR %
%\maketermo      % cria folha com o termo de aprovacao da dissertacao%

%\singlespacing           % espacamento 1 - capa UFPR%
%\onehalfspacing          % espacamento 1/2 %
\doublespacing            % espacamento 2 - UFPR %

\pagestyle{headings}
\pagenumbering{roman}

%\chapter*{Agradecimentos}
%\input{agradecimentos.tex}          % possiu somente o texto

\tableofcontents

%\listoffigures         % se houver mais do que 3 figuras
%\addcontentsline{toc}{chapter}{\MakeUppercase{Lista de Figuras}}
%\newpage

%\listoftables        % se houver mais do que 3 tabelas
%\addcontentsline{toc}{chapter}{\MakeUppercase{Lista de Tabelas}}
%\newpage

\chapter*{Resumo}
\addcontentsline{toc}{chapter}{\MakeUppercase{Resumo}}
%This paper presents a summary, which aims to cover the main ideas available on the
%literature of the past few years regarding network intrusion detection. The goal is to
%explain the general architecture of Network Intrusion Detection Systems (NIDS) and how
%different machine learning algorithms are applied. The focus of this review will be the
%anomaly-based systems, which present a vast field of study.
%
%
%\noindent \textbf{key-words}: nids, network security, intrusion detection, machine learning, data mining, pattern recognition

Este trabalho de graduação tem como intuito apresentar um estudo acerca da usabilidade e viabilidade de
algoritmos de \textit{data mining} e \textit{machine learning} no contexto de sistemas de detecção de
intrusão em rede. Sua utilização teria grande impacto na segurança tanto de servidores quanto de clientes,
pois promete evitar ataques ao sistema detectando anomalias na rede.
Neste artigo são apresentadas técnicas de \textit{machine learning}; a arquitetura geral
de um sistema baseado em anomalia; o processo de extração e processamento de dados; e, por fim, demonstra,
os impactos de uma base de dados representativa, de tipos de ataques desconhecidos e da seleção de atributos.
Os estudos sugerem que a imprevisibilidade de novos tipos de ataques ainda é um empecílio, mesmo para
sistemas com capacidade de aprender ao ser alimentado com novos dados.

\noindent \textbf{Palavras-chave}: NIDS, network security, intrusion detection, machine learning, data mining,
pattern recognition, segurança, aprendizado de máquina, mineiração de dados
           % somente o texto
\newpage


\pagenumbering{arabic}

\chapter{Introdução}
\label{ch:introducao}

Ao longo da última década, a realidade da comunicação em rede entre computadores pessoais e servidores evoluiu
para uma grande parcela da população, deixando de servir apenas para compartilhamento de websites.
Tornou-se um completo ambiente utilizado por cidadãos para armazenamento de arquivos pessoais na núvem;
por bancos para efetuar transações monetárias; e por governos para trocas de informações preciosas.
Assim como é possível se conectar de uma máquina pessoal a um servidor \textit{web}, é possível fazer o contrário.
Qualquer máquina conectada a uma rede está teoricamente sujeita ao acesso de fora.
Roteadores e sistemas operacionais atuais estão equipados com ferramentas básicas para previnir acesso não autorizado
ao computador. Ainda assim, há pessoas dedicadas a burlar tais proteções. Alguns desses invasores conseguem copiar
arquivos, instalar \textit{malware} e até ganhar controle sobre o sistema sem serem notados pelo dono.
\par A fim de cobrir esse ponto cego, foram criados sistemas de detecção de intrusão, ou IDS
(\textit{intrusion detection systems}). Seu objetivo é reconhecer comportamentos incomuns dentro de uma rede.
No meio acadêmico, muitos artigos foram publicados explorando algoritmos de Machine Learning nesta aplicação.
\par Existem diversos tipos de ataques e diferentes protocolos de comunicação, mas o problema de se detectar uma
intrusão pode ser simplificado em um problema de classificação binária do comportamento do sistema, onde os dois
possíveis resultados seriam "normal" ou "intrusivo". Existem duas "escolas de pensamento" bem definidas no meio
acadêmico acerca do assunto:
\begin{enumerate}
    \item \textit{Signature-based} (baseado em assinatura) -- Esse é o método padrão e atualmente o único utilizado
    comercialmente. Ele possui baixa adaptabilidade a novas ameaças.
    \item \textit{Anomaly-based} (baseado em anomalia) -- Ainda uma área de estudo, possui alta taxa de
    Falso Positivo, mas boa adaptabilidade e será o foco deste trabalho.
\end{enumerate}

\chapter{Análise de Trabalhos Relacionados}
\label{ch:analise}

NIDS baseado em anomalia tem se mostrado uma área de estudo promissora, mas ainda possui grande margem para pesquisa e melhorias.
A fim de analisar o estado da arte, considerei o estudo feito em \cite{tsai09}, que traz dados de 55 artigos publicados
entre os anos 2000 e 2007. A partir dessa análise, pude extrair algumas inferências sobre as diferentes técnicas utilizadas.
\begin{itemize}
    \item  26 se baseiam no uso de classificadores unitários, apesar de que o uso de
    classificadores híbridos tem aumentado durante os últimos anos.
    Classificadores compostos são usados por poucos, totalizando apenas 6.

    \item Os algoritmos unitários mais estudados para detecção de instrusão foram SVM (7) e K-NN (6).

    \item Quanto a classificadores híbridos, o método \textit{integrado} é o mais popular, consistindo de 10 dos 23.

    \item A utilização de seleção de atributos tem crescido com o tempo. Enquanto em 2003 apenas um quarto dos estudos analisados
por \cite{tsai09} utilizava tal técnica, em 2007 foi o caso de 9 dentre 11. Ou seja, apenas dois (\cite{peddabachigari07}, \cite{li07}) não utilizaram.

    \item Como há poucas bases de dados públicas para detecção de instrusão, a maioria dos estudos se baseia nas
    bases KDD Cup 99 (30) e DARPA 1998 (18).
\end{itemize}

\par Quanto aos dados, as técnicas pioneiras costumavam criar modelos considerando todo o corpo do pacote. Entretanto,
eles separavam as instâncias com base no tamanho do pacote e portas de destino, diminuindo o contexto de cada modelo.
Técnicas posteriores levam em consideração partes específicas do conteúdo do pacote. Aparentemente, anomalias mais
sutis podem ser detectadas a uma taxa menor de Falso Positivo quando usando um contexto mais específico.
\par Dois artigos recentes trouxeram implementações promissoras a respeito de sistemas de detecção de
anomalia. Um deles (\cite{lin12}) apresenta um algoritmo inteligente e o outro (\cite{papadonikolakis12}) mostra o
ganho de desempenho ao se programar o sistema diretamente em uma placa FPGA. Para os fins desta monografia, não irei
implementar o algoritmo em FPGA e, apesar de o foco não ser classificadores em cascata, os estudos relacionados merecem
notoriedade por consituirem o estado da arte no que se propõem a fazer.

\section{Processando grandes quantidades de dados}
A simulação apresentada em \cite{lin12} usou a base do KDD Cup 99. O algoritmo funciona pré-processando os dados de treino e teste e gerando uma solução
inicial aleatória. A solução é atualizada a cada iteração do algoritmo. Enquanto um limite não é satisfeito,
são utilizados Support Vector Machine (SVM) e Simulated Annealing (SA) para selecionar o melhor conjunto de
características. Por conseguinte, Árvore de Decisão e SA são usados para aumentar a precisão do teste e construir
regras de decisão. No final, a melhor precisão e as melhores características e regras de decisão são anunciadas.
\par Nos testes descritos no  artigo, após a seleção de atributos, foram utilizadas 23 características durante o processo de
classificação. O algoritmo conseguiu atingir uma precisão de 99,6\%, o que representa um grande ganho quando comparado
 a outros resultados até o momento.

%%%%%%%%%%%%%%%%%%%%%%%%%%%%%%%%%%%%%%%%%%%%%%%%%%%%%%%%%%%%%%%%%%%%%%%%%%%%

\section{Aumentado a velocidade de processamento}
Enquanto o algoritmo anterior pretende otimizar apenas a precisão, o estudo apresentado em \cite{papadonikolakis12}
foca na aceleração do processo de classificação. Aqui também foi utilizado SVM para implementar o A-NIDS, mas dessa
vez diretamente em um FPGA, fazendo uso de seu potencial de processamento paralelo.
\par Foram propostas três arquiteturas diferentes.
\begin{enumerate}
    \item \textit{Heterogeneous Baseline Classifier} -- O objetivo é explorar os requerimentos de precisão dos atributos. Ele
    tenta maximizar paralelização enquanto mantém a relação entre blocos de processamneto digital e blocos lógicos.
    \item \textit{Fit Cascade Chain} -- Aproveita o potencial de aritmética customizada oferecido pelo FPGA para criar uma
    cascata com duas precisões aritméticas diferentes. Primeiro um classificador de baixo custo e alta vazão com baixa
    precisão é usado. Somente os dados que não puderem ser classificados são alimentados ao segundo classificador. Este
    possui alta precisão e, logo, maior custo, mas com menos requisições de vazão.
    \item \textit{Reconfigurable Cascade Chain} -- Foi projetado para problemas com grandes quantidades de dados, que podem
    exceder as limitações de hardware. Essa arquitetura propõe uma reconfiguração do FPGA depois do estágio de treino e
    expande o espaço de design possível. Isso permite a utilização de mais blocos durante as classificação.
\end{enumerate}
\par A base utilizada para testes foi a \textit{MINIST Dataset}. Os autores haviam previamente executado testes similares em \emph{GPU} (unidade de processamento gráfico).
Comparando com testes prévios, o design \textit{Heterogeneous Baseline Classifier} processou os dados 7 vezes mais rápido. Os melhores resultados foram
obtidos usando \textit{Reconfigurable Cascade Chain}, com um ganho de 20 vezes, isto é, levando apenas 5\% do tempo
que a GPU consumiu.
\chapter{Signature-based NIDS}
\label{ch:snids}
Apesar de não ser o foco desta monografia, vamos descrever brevemente os sistemas baseados em assinatura, a fim de
comparação com os métodos nos quais vamos nos aprofundar.
\par Sistemas baseados em assinatura têm sido os mais bem sucedidos para detecção de intrusão até hoje.
A ideia é comparar o tráfego da rede com padrões de comportamento conhecidos durante certos ataques.
O maior problema é a ineficiência em detectar ataques novos.
\par Os primeiros sistemas utilizavam apenas o método de reconhecimento de padrão. Essa técnica possui um banco de dados
 preenchido com assinaturas de cada ameaça conhecida. Um evento malicioso é detectado se o atual estado da rede se iguala
 a uma assinatura. Como ele tenta comparar todas as assinaturas e a quantidade de dados transmitidos simultaneamente
cresce a cada ano, o custo computacional tronou-se muito alto.
\par Uma segunda técnica é o método de \textit{implication rules}, ou regras de inferência. Ela fornece um conjunto de
regras que descrevem eventos conhecidos que podem inferir o acontecimento de uma intrusão. De qualquer modo, IDS de
assinatura geralmente necessita de um humano capacitado, que cria um novo conjunto de regras toda vez que um novo
tipo de ataque surge, para adquirir modelos de tráfego.
\par Daí surge a motivação para se utilizar técnicas de \textit{data mining} em sistemas de assinatura. Elas proveem um
 modo de aumentar a automação no momento de construção e de ajuste do modelo. Eles podem adaptar os modelos à medida que
 acessam tráfego de rede contendo novas ameaças ou detectando diferentes versões de um ataque conhecido. Ainda assim,
 é impossível identificar comportamentos maliciosos completamente desconhecidos.
\chapter{Anomaly-based NIDS}
\label{ch:anids}
 A técnica de detecção de anomalia analisa o atual comportamento da rede para checar se corresponde ou não a um
 comportamento normal. O maior benefício desse método é a habilidade de identificar novos ataques com sucesso.
 A desvantagem é a alta taxa de Falso Positivo. Ao invés de requerer um novo modelo a ser adaptado, sistemas baseados
 em anomalia necessitam de dados de tráfego de rede livres de ataques. \textit{Data mining} também tem sido
 utilizado em sistemas baseados em anomalia, junto a outros métodos estatísticos e de \textit{machine learning}
 (aprendizado de máquina).
 \par As diferentes técnicas podem ser agrupadas em três categorias, de acordo com o processamento envolvido:
 estatístico, knowledge-based (baseado em conhecimento) ou algoritmos de aprendizado de máquina.

\section{Técnicas Estatísticas}
Um modelo de probabilidade de um determinado comportamento é criado a partir da atividade capturada do tráfego de rede,
 tal como taxa de tráfego, número de pacotes e quantidade de endereços de IP distintos.
 \par Ele não requer conhecimento prévio pois está capacitado a aprender o comportamento normal a partir de observações
 sem ataques. Outro benefício das técnicas estatísticas é a possibilidade de identificar, com precisão, atividades
 maliciosas que ocorrem ao longo de períodos de tempo mais extensos. Por outro lado, nem todos os possíveis
 comportamentos conseguem ser modelados e o equilíbrio entre taxas de Falso Positivo e Falso Negativo dependem
 fortemente na configuração correta dos parâmetros.

 \section{Técnicas Baseadas em Conhecimento}
 Também chamados de \emph{expert systems}, sistemas baseados em conhecimento são implementados criando-se um conjunto de
 regras de classificação para categorizar os dados. O modelo é geralmente criado por um humano experiente no campo
 da aplicação. Esse tipo de sistema não aponta novas atividades inofensivas como sendo maliciosas, garantindo, assim,
 um número reduzido de Falso Positivo. Por essa razão, o conjunto de regras precisa ser específico o suficiente e,
 apesar de ser possível atingir um certo nível de automação usando uma máquina de estados finitos, requer
 grande conhecimento sobre o comportamento da rede e tempo significativo para ser desenvolvido.

 \section{Técnicas de Machine Learning}
 Métodos de aprendizado de máquina exigem um conjunto de dados categorizados para treinar o modelo do comportamento
 esperado. A característica principal dessa técnica é a habilidade de adaptar suas regras de classificação à medida que
 novos dados são recebidos. Isso garante pouca necessidade por intervenção humana, uma vez funcionando. O lado ruim
 do aprendizado de máquina é o alto custo computacional. Os algoritmos mais populares usados em A-NIDS são:
 \begin{itemize}
    \item Rede Bayesiana -- cria uma rede de relações de probabilidade entre características.
    \item Modelos de Markov -- compara a probabilidade dos dados observados com um \emph{threshold} definido.
    \item Redes neurais -- cria uma rede de \emph{perceptrons} com habilidade de se adaptar.
    \item Lógica Fuzzy -- enxerga características como variáveis \emph{fuzzy} e classifica com base em valores contínuos.
    \item Algoritmos Genéticos -- deriva regras de classificação e seleciona as características mais discriminantes.
    \item Support Vector Machines -- encontra um hiperplano que eficientemente separa ambas as opções de classificação.
    \item Árvores de Decisão -- cria uma árvore onde cada folha representa uma classe e arestas correspondem a
        diferentes valores de atributos.
    \item Clustering -- algoritmo não-supervisionado que agrupa dados com base em suas similaridades.
\end{itemize}

\section{Combinação de Classificadores}
Como cada método possui seus pontos fortes e pontos fracos, uma ideia bastante estudada no meio acadêmico é a
 combinação de classificadores. Tais combinações podem se dar de diversas maneiras, como por exemplo: Classificadores
 Híbridos, Composição de Classificadores \footnote{Tradução livre do termo em inglês Ensemble Classifiers.} e
 Classificadores em Cascata.

 \subsection{Classificadores Híbridos}
    Um classificador híbrido consiste de dois componentes. O primeiro pré-processa o dado de entrada e envia ao segundo
    classificador, que chega ao resultado final. O primeiro componente de classificadores híbridos pode ser usado tanto
    como uma técnica de clustering para achar as classes que o segundo componente utilizará para categorizar os dados;
    quanto como um otimizador de performance para o segundo modelo, o que corresponderia ao método \textit{integrado}
    \cite{aydin09}.


 \subsection{Classificadores em Cascata}
    Pode ser considerado uma extensão dos classificadores híbridos, podendo consistir de inúmeros classificadores, onde
    o n-ésimo classificador utiliza como entrada os dados rejeitados pelo classificador anterior, ou seja, que não
    atingiram um grau de certeza satisfatório. Alguns estudos sugerem otimização de \emph{thresholds} para aumentar precisão
    \cite{oliveira05}.

 \subsection{Composição de Classificadores}
    Uma composição de classificadores pode ser obtida usando-se técnicas fracas de aprendizado
    (geralmente mais rápidas). Cada classificador é treinado usando um subconjunto distinto dos dados.
    A base de testes é, então, processada por todos eles e, finalmente, categorizados pela maioria.


\chapter{Sistemas A-NIDS funcionais}
    O estudo feito em \cite{teodoro09} apresenta sistemas A-NIDS divididos em duas categorias: comerciais e de pesquisa.
    Sistemas comerciais geralmente trabalham com um núcleo baseado em assinatura, enquanto sistemas de pesquisa
    desenvolvem protótipos e buscam metodologias inovadoras.

 \section{Arquitetura}
    Enquanto alguns detalhes de implementação podem variar, o padrão da esquematização básica de um A-NIDS
    é como apresentado em \cite{catania12}:
    \begin{itemize}
        \item Aquisição de dados de tráfego -- coleta informação sobre os quadros da rede para processamento futuro.
        \item Gerador de características do tráfego -- extrai características do tráfego capturado. Tais características
         podem ser classificadas como "baixo nível" (obtidas diretamente do dado bruto), "alto nível" (deduzidas de um
         processamento subsequente), "pacote" (coletadas de cabeçalhos de pacotes), "fluxo" (contendo informação das
         conexões) e "payload" (obtidos da carga do pacote).
        \item Detector de incidente -- identifica atividades intrusivas. Pode ou não conter um segundo núcleo baseado
        em assinatura..
        \item Gerador de modelo de tráfego -- contém informação base usada para guiar o detector de incidente.
        \item Gerenciamento de resposta -- inicia manobras para superar uma intrusão em potencial. É inicializado pelo
        detector de incidente.
    \end{itemize}

 \section{Comercialmente}
 Um dos primeiros projetos de detecção de anomalia a ser amplamente conhecido foi o SPADE, um plugin para o Snort que
 analisava transferência de pacotes procurando comportamentos fora do comum. Alternativamente, Stealthwatch utilizava
 detecção de anomalia baseada em fluxo.
 \par Sistemas recentes usam uma arquitetura distribuída utilizando sensores e um
 console central para gerenciar o processo de detecção. Esse é o caso do DeepSight, que usa um método estatístico.
 A maioria dos sistemas comerciais utilizam um módulo de detecção baseado em assinatura aliado com um núcleo baseado em
 anomalia. Todos esses sistemas se enquadram na categoria de classificadores híbridos.
\chapter{Dados}
\label{chap:dados}
Os passos mais relevantes para o modelamento de dados de um NIDS são os seguintes:
\begin{enumerate}
    \item Criação da base -- identifica dados categorizados (normal ou anômalo) representativos para treino e teste.
    Categorização de tráfego de rede pode ser uma tarefa difícil e longa, que geralmente envolve um especialista.
    \item Construção de características -- cria características com maior discriminabilidade. Tais características
    podem ser construídas por um humano ou por algoritmos de aprendizado de máquina.
    \item Redução -- também chamado de "seleção de caterísticas", diminui a dimensionalidade da base de dados
    descartando características irrelevantes ou redundantes. Usado para atenuar a "Maldição da Dimensionalidade"
    \footnote{\textit{Curse of Dimensionality}: Ao se usar muitas características,
    supostamente se obtém baixa precisão.}.
\end{enumerate}

%%%%%%%%%%%%%%%%%%%%%%%%%%%%%%%%%%%%%%%%%%%%%%%%%%%%%%%%%%%%%%%%%%%%%%%%%%%%%%%%%%%%%%%%%%

\section{Origem dos Dados}
A escolha de informações da rede é amplamente afetada pelos requisitos de detecção ao se projetar o sistema.
É possível ter sistemas de detecção específicos, utilizando um conjunto de características limitado.
Para um sistema mais genérico, o ideal seria utilizar detectores separados utilizando conjuntos de características
distintos, um para cada especificidade. Dependendo da origem desses dados, têm-se algumas vantagens e desvantagens.
Técnicas para análise de conteúdo ainda não estão tão concretizadas quanto as que extraem características de
cabeçalhos. Assim comoanalisar conteúdo do lado do cliente ainda é um campo não tão estudado quanto o lado do servidor
numa perspectiva de A-NIDS. Técnicas do lado do cliente almejam detectar ameaças em aplicações web, como \textit{drive-by
downloads}, \textit{cross-site scripting} e trechos maliciosos de JavaScript.

\subsection{Cabeçalho de pacote}
Características obtidas através de  cabeçalhos de pacotes têm a qualidade de serem rápidos, sem exigir muito
processamento ou memória, e evitarem preocupações legais acerca de análise de dados da rede.
O conjunto de característica mais simples contém características básicas extraídas dos cabeçalhos. Essas
características podem ser usadas para apontar pacotes individuais que são anômalos quando comaprados ao modelo de
treino; ou como um processo de filtragem para que apenas pacotes incomuns sejam usados por algoritmos de detecção
posteriores.
Entretanto, pacotes individuais não podem ser usados para identificar padrões incomuns durante um grande período.
Existem ataques que contém cabeçalhos normais quando analisados individualmente, enquanto sua repetição durante
um certo tempo pode ser considerada anômala. Um exemplo seria o ataque de negação de serviço, popularmente conhecido
como \textit{DoS}.

\subsection{Corpo de pacote}
Quando ataques são destinados a aplicações, os bytes maliciosos estão dentro do corpo do
pacote e, portanto, as técnicas baseadas em cabeçalho não podem ser usadas. Isso representa um defeito considerável,
principalmente porque diversos ataques da atualidade não são direcionados à rede em si, mas a aplicações conectadas a
ela.
\par NIDS devem utilizar características baseadas em conteúdo, extraídas do corpo dos pacotes, para detectar tais tipos 
de ataques, uma vez que cabeçalhos podem aparentar completamente normais. Análise de conteúdo é computacionalmente 
mais cara do que análsie de cabeçalho porque requer uma inspeção mais profunda do pacote. Ela lida com uma variedade de 
tipo de conteúdo (pdf, jpg, HTML), compressão, e métodos que encobrem dados. Entretanto, o benefício da análise do
corpo é ter acesso a todos os bytes transferidos entre os aparelhos na rede, permitindo a cosntrução de um rico
conjunto de características baseadas em conteúdo para detecção de anomalia.
\par Como análise de conteúdo possui uma alta complexidade, diversos métodos focam em pequenos subconjuntos de
conteúdo, como requisições HTTP ou apenas o JavaScript de um conteúdo baixado. Métodos baseados em anomalia não tentam
comparar assinatoras de malware conhecido, mas podem aplicar heurísticas, como Casamento de Padrões para detectar a
presença de código shell.

%%%%%%%%%%%%%%%%%%%%%%%%%%%%%%%%%%%%%%%%%%%%%%%%%%%%%%%%%%%%%%%%%%%%%%%%%%%%%%%%%%%%%%%%%%

\section{Conjuntos de Características}
\subsection{Conexões múltiplas}
Grande parte dos NIDS analisados em \cite{davis11} usam dados de rede referentes a fluxo ou sessão. Características
são construídas a partir do fluxo. O método mais popular é o de cabeçalho de pacotes utilizando características
derivadas de múltiplas conexões (MCD). Essas características são geralmente divididas usando média, desvio padrão, e
procentagem de fluxos, cobrindo múltiplas sessões. NIDS baseados em anomalia que utilizam essas caracterśiticas são
capazes de discernir entre atividade normal da rede e tráfego incomum como \textit{DoS} e \textit{scanning}.
\par características MCD são geralmente extraídas a partir de conexões dentro de um intervalo de tempo. A maioria das
 características de MCD são baseadas em volume, como o a quantidade de conexões a um endereço de IP e porta
 em um espaço de tempo. Portanto, características derivadas de múltiplas conexões podem ser facilmente usados para detectar
 volumes excessivos de tráfego relacionados a \textit{DoS} e \textit{probing}. Como pacotes anômalos individuais não
 suprem o valor baseado em volume, eles podem ser ignorados.

\subsection{Única conexão}
Características derivadas de conexões individuais (SCD) são utilziadas para detectar comportamento anômalo dentro de
uma úncia sessão. Elas podem apontar um protocolo inesperado, tamanhos incomuns de dados, timestamp não condizente,
ou sequências incomuns de \textit{flags} de TCP. Portanto, características SCD permitem detecção de uso anômalo da rede
por \textit{backdoors}, túnel HTTP e afins.
\par As características SCD fornecem contexto que pode ser usado para encotnrar anomalias não contextuais. Por exemplo, se a
temporização de pacotes dentro de uma porta monitorada não se encaixa com um perfil esperado, um alerta pode ser
disparado, uma vez que pode se tratar de um protocolo de tunelamento.

\subsection{KDD Cup 99}
O KDD Cup 99 se trata de uma base dados para uso livre. Ela possui 32 características quantitativas e
9 qualitativas, junto a tráfego normal e ataques de \textit{probing}, negação de serviço, \textit{user to root} e
\textit{remote to local}.  Tais características foram construídas por um especialista e involvem informações de mais
alto nível, como o número de tentativas de login sem sucesso, se acesso root ao terminal foi conseguido,
e o número de operações de criação de arquivo.
\par Consdierando que o KDD Cup 99 já tem mais de 15 anos, é provável que essas características não sejam mais úteis
para detectar os tipos de ataques recentes na realidade do fluxo de dados na rede atualmente. Para se detectar
ameaças atuais, é necessário construir características baseadas em conteúdo novas, uma vez que o conteúdo em si mudou.

%%%%%%%%%%%%%%%%%%%%%%%%%%%%%%%%%%%%%%%%%%%%%%%%%%%%%%%%%%%%%%%%%%%%%%%%%%%%%%%%%%%%%%%%%%

\section{Pré-processamento}
Como anteriormente apontado, sistemas baseados em anomalia estão sujeitos a uma alta taxa de Falso Positivo. Apenas
1\% sequer de Falso Positivo resulta em um número absurdo de falsos alertas. Deve-se ter em mente que servidores lidam
com centenas de conexões simultâneas e inúmeros pacotes a cada centésimo de segundo.
\par O pré-processamento de dados é, então, exigido para atingir uma melhor performance na detecção de intrusão.
O pré-processamento converte tráfego de rede em uma série de ocorrências, onde cada uma é representada por um vetor
de características. As informações seguintes acerca dos dados e características foram baseadas nos estudos presentes
em \cite{davis11}.

\subsection{Derivação}
Dfierentes tipos de características que podem ser usados em NIDS baseados em anomalia foram cobertos em \cite{davis11}.
Cada tipo de característica é derivada de vários métodos de pré-processamento, incluindo organizar pacotes em fluxos,
analizar conteúdo de aplicações em busca de campos de interesse ou percorrer cabeçalhos de cada pacote. A derivação do
potencial conjunto de características é um passo essencial para o NIDS baseado em anomalia. Analogamente,
pré-processamentos subsequentes podem ajudar ainda mais a aumentar a eficiência do NIDS.
\par Métodos de pré-processamento de mineiração de dados podem ser usados, incluindo tranformação, redução e
discretização de dados, como os apresentados em \cite{ribeiro08}. Uma técnica de redução geralmente utilziada é a
\textit{Principal Component Analysis}. PCA tem se mostrado útil para redução da dimensionaldiade dos dados,
consequentemente redizindo o custo computacional do sistema de detecção
\par Para eliminar características redundantes, diversos algoritmos automatizados para seleção de características
também existem, resultando em redução de dados similar. Tais métodos de redução de dados fornecem um meio preciso para
a conversão de um conjunto candidato para o conjunto final de características. É possível que métodos automáticos de
redução possam melhorar ainda mais o NIDS, apesar de que alguns sistemas são construídos apenas se baseando no
conhecimento de um especialista para construir conjuntos de características significativos.

\subsection{Alternativas de processamento}
Considerando métodos que almejam maior automatização na momento do processamento dos dados e características, existem
duas ferramentas aparentemente promissoras: N-grams e libAnomaly. N-grams pode ser usado com algoritmos de seleção
automática de características, enquanto o libAnomaly se trata de uma biblioteca livre para modelos de pré-processamento
de dados.
\par No N-grams, perfis são construídos para diferentes tipso de arquivos, calculando as distribuições de frequência
de byte. O NIDS compara, então, o tráfego observado com os perfis usando Distância Manhattan. N-gram é geralmente
usado para analsiar requisições ao servidor. É útil em detectar padrões anômalos, como código shell dentro de
protocolos estruturados de aplicações, sem exigir supervisão muita supervisão.
\par O foco do libAnomaly, por sua vez, é construir um punhado de modelos representando a transferência normal de
dados entre usuário e aplicação web. Requisições maliciosas provavelmente serão apontadas como anômalas quando
comparadas com um ou mais modelos, pois geralmente se diferem de requisições normais de alguma forma.
\par Embora a intenção é que NIDS sejam completamente automáticos, sua criação e adaptação requerem um especialista.
Mesmo o N-grams necessita de auxílio em algumas áreas da rede. LibAnomaly também requer experiência significativa
para selecionar quais campos dentro da rede devem ser modelados. Por outro lado, detecção usando cabeçalhos de
pacotes requerem menos conhecimento. Técnicas de mineiração de dados podem ser usados para derivar características de
cabeçalhos MCD para detectar algumas ameaças. O conjunto de características em potencial  poderá, então, ser processado no
estágio de seleção de característica. Essas técnicas automatizadas garantem que as características disponíveis mais discriminantes sejam
escolhidas.

\chapter{Testes}
Três baterias de testes foram conduzidas ao longo da produção deste trabalho de graduação a fim de ver na prática o
impacto dos algoritmos de aprendizado de máquina quando aplicados em bases de dados com informações de rede.
Em um dos testes, a base utilizada foi a supracitada KDD Cup 99. Para os outros, utilizou-se uma série de dados
 gerados pelo mestrando da PUC-PR Eduardo Viegas, orientado pelo Prof. Dr. Altair Santin. Em ambos os casos, os algoritmos foram
 executados através do Weka.

\section{Weka}
O \textit{Waikato Environment for Knowledge Analysis} (Weka) é uma suíte de aplicações para aprendizagem de máquina e
mineiração de dados. O software foi desenvolvido em Java na \textit{University of Waikato} e lincenciado sob a
\textit{GNU General Public License}. Os destaques dessa ferramenta incluem a vasta quantidade de algoritmos
implementados, pré-processamento de dados, seleção de atributos e visualização de dados. O Weka funciona como uma
interface, permitindo que novos algoritmos em Java sejam produzidos e facilmente executados de forma gráfica e com
resultados interativos \cite{bouckaert10}.

\section{Avaliação de Performance}
Gráficos ROC (\textit{Receiver Operating Characterístics}) são comumente usados como um método  para visualizar
a performance de classificadores e analisar as taxas de acertos e alarmes falsos \cite{fawcett04}. Um gráfico ROC
baseia-se em apenas duas classes: positivo ou negativo. Em termos de detecção de intrusão, uma conexão pode ser
considerada inofensiva (positivo) ou prejudicial (negativa). Para avaliar a precisão de um classificador,
duas outras categorias podem ser usadas: verdadeiro e falso. Quando cruzadas, obtém-se quatro possíveis resultados,
como mostra a tabela \ref{tab:tfpn}.

\begin{table}[h]
    \centering
    \caption{Categorias para análise de corretude}
    \label{tab:tfpn}
    \begin{tabular}{l|l|l|}
        \cline{2-3}
                                                               & \cellcolor[HTML]{EFEFEF}Verdadeiro (T)                                                           & \cellcolor[HTML]{EFEFEF}Falso (F)                                                     \\ \hline
        \multicolumn{1}{|l|}{\cellcolor[HTML]{EFEFEF}Positivo} & \begin{tabular}[c]{@{}l@{}}TP -- Conexão inofensiva é \\ classificada corretamente\end{tabular}  & \begin{tabular}[c]{@{}l@{}}FP -- Conexão anômala é \\ considerada normal\end{tabular} \\ \hline
        \multicolumn{1}{|l|}{\cellcolor[HTML]{EFEFEF}Negativo} & \begin{tabular}[c]{@{}l@{}}TN -- Conexão prejudicial é \\ classificada corretamente\end{tabular} & \begin{tabular}[c]{@{}l@{}}FN -- Conexão normal é \\ considerada anômala\end{tabular} \\ \hline
    \end{tabular}
\end{table}

Quando se sabe a verdadeira natureza dos casos de teste, é possível determinar a precisão, a taxa de Verdadeiro Positivo
 ($TP_{rate}$) e taxa de Falso Positivo ($FP_{rate}$). É possível inferir que uma das maiores preocupações ao se construir
 um A-NIDS é a minimização a taxa de Falso Positivo. Teoricamente se 100\% das conexões forem consideradas anômalas
 e bloqueadas, teremos $FP_{rate}$ igual a 0. Por um lado, o sistema estaria bloqueando conexões inofensivas, por outro,
 estaria com certeza evitando conexões ameaçadores. O cálculo das taxas se dá da seguinte maneira.

 $$ Precis\tilde{a}o = \frac{TP + TN}{P + N} $$
 $$ TP_{rate} = \frac{TP}{P} $$
 $$ FP_{rate} = \frac{FP}{N} $$

Essas taxas estão contidas no espaço entre $0$ e $1$, inclusivo, e são utilizadas para plotar o ROC. A $TP_{rate}$ é
usado no eixo $y$ e a $FP_{rate}$, no eixo $x$. Portanto, o gráfico ROC elicita a relação entre Verdadeiro Positivo e
Falso Positivo.
\par A Figura \ref{fig:roc} mostra uma curva ROC simples com 5 classificadores fictícios: A, B, C, D e E. Cada classificador
retorna um único ponto no espaço ROC. Se o ponto estiver localizado na coordenada (0,0), como o classificador A na
Figura \ref{fig:roc}, significa que nenhum caso foi considerado inofensivo. Classificadores que se encontram próximos ao eixo
$y$ podem ser considerados "conservadores", uma vez que necessitam de forte evidência para categorizar uma ocorrência
como positiva. Em contrapartida, o ponto B, posicionado em (1,1), representa sistemas que não acusariam nenhuma
conexão como anômala. Classificadores mais à direita do gráfico podem ser considerados "liberais" pois presumem como
positivo mais facilmente.

\begin{figure}
  \centering
    \begin{tikzpicture}
        \begin{axis}[
            xlabel={$FP_{rate}$},
            ylabel={$TP_{rate}$},
            xmin=0, xmax=1,
            ymin=0, ymax=1,
            ymajorgrids=true,
        ]
            \addplot[
                color=black!30,
                ]
                coordinates {
                (0,0)(1,1)
                };

            \addplot [ color=blue, mark=*, nodes near coords=A, every node near coord/.style={anchor=180}]
                coordinates {( 0, 0)};
            \addplot [ color=blue, mark=*, nodes near coords=B, every node near coord/.style={anchor=180}]
                coordinates {( 1, 1)};
            \addplot [ color=blue, mark=*, nodes near coords=C, every node near coord/.style={anchor=180}]
                coordinates {( 0, 1)};
            \addplot [ color=blue, mark=*, nodes near coords=D, every node near coord/.style={anchor=180}]
                coordinates {( 0.2, 0.8)};
            \addplot [ color=blue, mark=*, nodes near coords=E, every node near coord/.style={anchor=180}]
                coordinates {( 0.8, 0.2)};

        \end{axis}
    \end{tikzpicture}
    \caption{Exemplo da distribuição de 5 classificadores}
    \label{fig:roc}
\end{figure}

\par Um classificador perfeito resultaria no ponto (0,1), onde C se encontra. Isso significaria possuir $FP_{rate}$ de
0\% e $TP_{rate}$ de 100\%, ou seja, todas as conexões intrusivas serem detectadas e nenhuma conexão normal ser
bloqueada pelo sistema. Assim sendo, quanto maior o $TP_{rate}$ e menor o $FP_{rate}$, mais preciso é o algoritmo.
 Se um classificador está localizado abaixo da linha diagonal ($y = x$), isto significa que sua performance é pior do
 que a escolha aleatória de classes. O triângulo inferior direito é geralmente encontrado vazio pois em casos de
 classificação binária com alta taxa de erro, é possível simplesmente inverter os resultados. Portanto, E (0.8,0.2)
 seria tão preciso quanto D(0.2,0.8)


\section{Impacto de modelos representativos}
O primeiro teste realizado levou em consideração apenas o SVM padrão fornecido pelo libSVM para o Weka. A finalidade
era verificar o impacto da representabilidade da base de treino \cite{yaman11}, analisando a performance do algoritmo
quando treinado com bases de diferentes tamanhos. Para esse estudo, o KDD Cup 99 foi divido em 5 particionamentos,
cada um com tamanhos distintos de bases de treino e de testes, usando a opção de \textit{cross-validation} do Weka.
O comando executado incluiu um pré-processamento de heurística de encolhimento, peso $1$, \emph{seed} de $3$, erro de
$0.001$ e função núcleo \textit{Radial Basis}.

\begin{table}[h]
    \centering
    \caption{Comparação entre particionamentos de tamanhos diferentes}
    \label{tab:partic}
    \begin{tabular}{l|l|l|l|l|}
        \cline{2-5}
                                                        & \multicolumn{1}{c|}{\cellcolor[HTML]{EFEFEF}Tamanho Treino} & \multicolumn{1}{c|}{\cellcolor[HTML]{EFEFEF}Tamanho Teste} & \multicolumn{1}{c|}{\cellcolor[HTML]{EFEFEF}Tempo} & \multicolumn{1}{c|}{\cellcolor[HTML]{EFEFEF}Precisão} \\ \hline
        \multicolumn{1}{|l|}{\cellcolor[HTML]{EFEFEF}1} & 10\%                                                        & 90\%                                                       & 16h13m                                             & 92.22\%                                               \\ \hline
        \multicolumn{1}{|l|}{\cellcolor[HTML]{EFEFEF}2} & 20\%                                                        & 80\%                                                       & 24h02m                                             & 94.54\%                                               \\ \hline
        \multicolumn{1}{|l|}{\cellcolor[HTML]{EFEFEF}3} & 30\%                                                        & 70\%                                                       & 27h32m                                             & 95.65\%                                               \\ \hline
        \multicolumn{1}{|l|}{\cellcolor[HTML]{EFEFEF}4} & 40\%                                                        & 60\%                                                       & 32h11m                                             & 98.62\%                                               \\ \hline
        \multicolumn{1}{|l|}{\cellcolor[HTML]{EFEFEF}5} & 50\%                                                        & 50\%                                                       & 38h45m                                             & 99.64\%                                               \\ \hline
    \end{tabular}
\end{table}

\par A \ref{tab:partic} mostra os as proporções de base de testes e treino em cada particionamento e seus respectivos
resultados. É possível notar uma clara correlação entre o tamanho da base de treino e a porcentagem de acertos. O
melhor resultado foi de $99.64$\% quando metade da base foi usada para gerar o modelo. Entretanto, em uma situação real
não há dados disponíveis o suficiente para se representar metade das transmissões que ocorrem diariamente.


\section{Impacto da imprevisibilidade}
Uma abordagem recente direcionou os estudos desse trabalho para a análise da real capacidade de algoritmos de
aprendizado de máquinas em se adaptar a novas realidades e detectar ameaças totalmente desconhecidas \cite{sommer10}.
\par Através de simulações de rede, foi gerado um conjunto de dados, contendo conexões normais e anômalas,
dispostos em três grupos distintos, os quais chamaremos de \textit{Inicial}, \textit{Análogo} e \textit{Desconhecido}.
O primeiro é um limitado conjunto de conexões utilizando protocolos HTTP e SMTP. Este foi dividido em dois grupos:
um para treino, com $258880$ instâncias, e outro para testes, com $275204$. O grupo \textit{Análogo} possui dados de
$148343$ conexões e, como o nome sugere, possui ataques semelhantes ou derivados daqueles no grupo anterior.
Já o terceiro engloba, também, ataques completamente desconhecidos provenientes de outros
protocolos, como SNMP e SSH, totalizando $276548$ ocorrências. Apenas o grupo \textit{Inicial} foi utilizado para
treino. O objetivo é observar o comportamento do sistema quando novos
ataques são recebidos, sejam eles derivados dos conhecidos ou completamente novos.
\par Tanto o SVM quanto DT foram utilizados nesta análise. As tabelas \ref{tab:dt} e \ref{tab:svm} mostram os resultados
em cada etapa do processo, em termos de taxa de Verdadeiro Positivo, taxa de Falso Positivo, tempo para classificar
todas as instâncias da respectiva base, precisão e porcentagem de instâncias classificadas corretamente. As taxas de
acerto evidenciam uma grande perda de confiabilidade conforme o surgimento de novos ataques, sendo a classificação da
base \textit{Desconhecido} praticamente tão eficiente quanto uma adivinhação aleatória.

\begin{table}[h]
    \centering
    \caption{Resultados de DT em cada base}
    \label{tab:dt}
    \begin{tabular}{l|l|l|l|l|l|}
        \cline{2-6}
                                                                   & \multicolumn{5}{c|}{\cellcolor[HTML]{EFEFEF}DT}        \\ \cline{2-6}
                                                                   & $TP_{rate}$ & $FP_{rate}$ & Tempo & Acertos & Precisão \\ \hline
        \multicolumn{1}{|l|}{\cellcolor[HTML]{EFEFEF}Inicial}      & 0.982       & 0.018       & 6.92s & 98.18\% & 0.982    \\ \hline
        \multicolumn{1}{|l|}{\cellcolor[HTML]{EFEFEF}Análogo}      & 0.912       & 0.856       & 3.73s  & 91.23\% & 0.885    \\ \hline
        \multicolumn{1}{|l|}{\cellcolor[HTML]{EFEFEF}Desconhecido} & 0.503       & 0.497       & 6.95s & 50.29\% & 0.751    \\ \hline
    \end{tabular}
\end{table}

\begin{table}[h]
    \centering
    \caption{Resultados de SVM em cada base}
    \label{tab:svm}
    \begin{tabular}{l|l|l|l|l|l|}
        \cline{2-6}
                                                                   & \multicolumn{5}{c|}{\cellcolor[HTML]{C0C0C0}SVM}          \\ \cline{2-6}
                                                                   & $TP_{rate}$ & $FP_{rate}$ & Tempo    & Acertos & Precisão \\ \hline
        \multicolumn{1}{|l|}{\cellcolor[HTML]{EFEFEF}Inicial}      & 0.971       & 0.029       & 15h13min & 98.43\%  & 0.984     \\ \hline
        \multicolumn{1}{|l|}{\cellcolor[HTML]{EFEFEF}Análogo}      & 0.902       & 0.845       & 8h12min  & 90.27\%  & 0.891    \\ \hline
        \multicolumn{1}{|l|}{\cellcolor[HTML]{EFEFEF}Desconhecido} & 0.52        & 0.484       & 15h29min & 50.78\%  & 0.753     \\ \hline
    \end{tabular}
\end{table}



\section{Impacto da seleção de atributos}
Tendo em vista o alto custo computacional para processar as 51 características iniciais, foi ponderada a viabilidade
de se utilizar uma seleção de atributos prévia para agilizar o processamento na tomada de decisões. Utilizando o
PCA, a quantidade de características foi reduzida para 39. Novos testes foram executados, seguindo o mesmo método da
análise anterior.

\begin{table}[h]
    \centering
    \caption{Resultados de DT em cada base com seleção de atributos}
    \label{tab:dtattrless}
    \begin{tabular}{l|l|l|l|l|l|}
        \cline{2-6}
                                                                   & \multicolumn{5}{c|}{\cellcolor[HTML]{EFEFEF}DT}        \\ \cline{2-6}
                                                                   & $TP_{rate}$ & $FP_{rate}$ & Tempo & Acertos & Precisão \\ \hline
        \multicolumn{1}{|l|}{\cellcolor[HTML]{EFEFEF}Inicial}      & 1.0         & 0.0         & 5.75s & 99.98\% & 1.0      \\ \hline
        \multicolumn{1}{|l|}{\cellcolor[HTML]{EFEFEF}Análogo}      & 0.848       & 0.429       & 3.1s  & 84.82\% & 0.914    \\ \hline
        \multicolumn{1}{|l|}{\cellcolor[HTML]{EFEFEF}Desconhecido} & 0.499       & 0.501       & 5.78s & 49.86\% & 0.414    \\ \hline
    \end{tabular}
\end{table}

\begin{table}[h]
    \centering
    \caption{Resultados de SVM em cada base com seleção de atributos}
    \label{tab:svmattrless}
    \begin{tabular}{l|l|l|l|l|l|}
        \cline{2-6}
                                                                   & \multicolumn{5}{c|}{\cellcolor[HTML]{C0C0C0}SVM}          \\ \cline{2-6}
                                                                   & $TP_{rate}$ & $FP_{rate}$ & Tempo    & Acertos & Precisão \\ \hline
        \multicolumn{1}{|l|}{\cellcolor[HTML]{EFEFEF}Inicial}      & 0.503       & 0.502       & 14h45min & 50.3\%  & 0.75     \\ \hline
        \multicolumn{1}{|l|}{\cellcolor[HTML]{EFEFEF}Análogo}      & 0.068       & 0.068       & 7h56min  & 6.78\%  & 0.937    \\ \hline
        \multicolumn{1}{|l|}{\cellcolor[HTML]{EFEFEF}Desconhecido} & 0.5         & 0.5         & 14h51min & 50.0\%  & 0.75     \\ \hline
    \end{tabular}
\end{table}

As tabelas \ref{tab:dtattrless} e \ref{tab:svmattrless} mostram que o resultado dos algoritmos nesse
cenário teve nenhum ganho significativo. Enquanto os atributos selecionados podem ser mais descritivos para os
ataques vistos na base \textit{Inicial}, talvez não sejam interessantes para identificar os ataques correntes.
 A figura \ref{fig:roctest} evidencia uma diminuição de Falso Positivo na base \emph{Análogo}
 quando usada a seleção de atributo, mas ainda sendo muito superior ao \textit{Inicial}. O comportamento
 ao se classificar a base totalmente desconhecida mostrou-se tão bom quanto escolha aleatória.

\begin{figure}[h]
  \centering
    \begin{tikzpicture}
        \begin{axis}[
        scale only axis,
        width=0.65\textwidth,
            xlabel={$FP_{rate}$},
            ylabel={$TP_{rate}$},
            xmin=0, xmax=1,
            ymin=0, ymax=1,
            ymajorgrids=true,
        ]
            \addplot[
                color=black!30,
                ]
                coordinates {
                (0,0)(1,1)
                };

            %  DT
            \addplot [ color=blue, mark=*, nodes near coords=I, every node near coord/.style={anchor=180}]
                coordinates {( 0.018, 0.982)};
            \addplot [ color=blue, mark=*, nodes near coords=A, every node near coord/.style={anchor=180}]
                coordinates {( 0.856, 0.912 )};
            \addplot [ color=blue, mark=*, nodes near coords=D, every node near coord/.style={anchor=180}]
                coordinates {( 0.497, 0.503 )};

            %  SVM
            \addplot [ color=blue, mark=triangle*, nodes near coords=I, every node near coord/.style={anchor=180}]
                coordinates {( 0.029, 0.971)};
            \addplot [ color=blue, mark=triangle*, nodes near coords=A, every node near coord/.style={anchor=180}]
                coordinates {( 0.845, 0.902)};
            \addplot [ color=blue, mark=triangle*, nodes near coords=D, every node near coord/.style={anchor=180}]
                coordinates {( 0.484, 0.52)};

            %  DT_attrless
            \addplot [ color=red, mark=*, nodes near coords=I, every node near coord/.style={anchor=180}]
                coordinates {( 0, 1)};
            \addplot [ color=red, mark=*, nodes near coords=A, every node near coord/.style={anchor=180}]
                coordinates {( 0.429, 0.848)};
            \addplot [ color=red, mark=*, nodes near coords=D, every node near coord/.style={anchor=180}]
                coordinates {( 0.501, 0.499)};

            %  SVM
            \addplot [ color=red, mark=triangle*, nodes near coords=I, every node near coord/.style={anchor=180}]
                coordinates {( 0.502, 0.503)};
            \addplot [ color=red, mark=triangle*, nodes near coords=A, every node near coord/.style={anchor=180}]
                coordinates {( 0.068, 0.068)};
            \addplot [ color=red, mark=triangle*, nodes near coords=D, every node near coord/.style={anchor=180}]
                coordinates {( 0.5, 0.5)};

        \end{axis}
    \end{tikzpicture}
    \caption{Comparativo dos algoritmos SVM (triângulo) e DT (círculo) em base sem e com seleção de
    atributo (azul e vermelho, respectivamente). I -- \textit{Inicial};  A -- \textit{Análogo};
     D -- \textit{Desconhecido}}
    \label{fig:roctest}
\end{figure}
\chapter{Conclusão}
\label{ch:conclusao}

Em suma, os algoritmos de aprendizado tentam automatizar o processo de reconhecer ameaças dentro da rede
através da detecção de conexões anômalas, dando luz aos Sistemas de Detecção de Intrusão em Rede baseados em Anomalia.
Neste trabalho, foram analisados os diferentes tipos de dados utilizados para a detecção de intrusões em rede e a
performance de algoritmos sugeridos em trabalhos relacionados. Também foram estudados os impactos de uma base de
treino suficientemente representativa da rede para predição de ataques desconhecidos pelo sistema.
\par O primeiro teste, usando apenas SVM, mostrou a necessidade de se gerar uma base de treino grande o suficiente, a
fim de se evitar amostras pouco representativas. Os resultados do segundo teste sugerem que, possivelmente, os A-NIDS
não tenham tanta escalabilidade com o surgimento de ameaças completamente desconhecidas e que, portanto, seriam
necessárias bases específicas para aplicações específicas ao invés de sistemas genéricos. O último teste tinha a
intenção de melhorar os resultados anteriores através da seleção de atributos, mas o seu ganho de performance não foi
significativo.
\par O campo ainda está aberto para novos estudos e análises quanto a viabilidade do uso de machine learning em tempo
real para a detecção de novas ameaças. Existem outros pontos não abordados nesse trabalho que também têm espaço para
serem otimizados, tais quais a atual falta de métricas unificadas; o processamento do alto, rápido e contínuo fluxo de
dados na rede; análise de dados cifrados e a taxa aceitável de de Falso Positivo. Por fim, pode-se considerar que
 o maior desafio nos testes de detecção de intrusão em rede reside em adquirir uma base de treino e de testes que
 represente precisamente o atual comportamento da rede.
% Trocar para ficar no padrão brasileiro
\bibliographystyle{brazil}
%\bibliographystyle{plain}

\inputencoding{latin2}
\bibliography{bib}
\inputencoding{utf8}
% utilize macros (3 primeiras letras do mes em ingles, minusculas) no seu
% .bib para atribuir o nome do mes em portugues nas referencia,
% se o style for brazil, outros estilos tambem aceitam estas macros
% Ex:
%
% @InProceedings{teste,
%   author =       {Luciano}
%   year =         {2000}
%   month =        {}#sep;
% }
%
\addcontentsline{toc}{chapter}{\MakeUppercase{Bibliografia}}

\singlespacing

\end{document}
