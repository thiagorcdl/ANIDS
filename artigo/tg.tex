\documentclass[12pt,a4paper]{ufpr}

\usepackage[brazil]{babel}
\usepackage[utf8]{inputenc}
\usepackage{amssymb,amsmath}
\usepackage{multirow}
\usepackage{amssymb}
\usepackage{subcaption}
\usepackage{graphicx}
\usepackage{caption}
\usepackage{setspace}
\usepackage{diagbox}

% n - numero de niveis de subsubsection numeradas
\setcounter{secnumdepth}{3}
% coloca ate o nivel n no sumario
\setcounter{tocdepth}{3}
\graphicspath{{img/}}

\title{Sistemas de Detecção de Intrusão em Rede baseados em Anomalias}
\author{Thiago Roscia Cerdeiro de Lima}
% ou Orientador
\advisortitle{Orientador}
\advisorname{Prof. Dr. Luis Allan Kunzle}
% departamento, instituicao
\advisorplace{Departamento de Informática, UFPR}
\city{Curitiba}
\year{2015}

% nao insira o nome do orientador, ja eh feito automaticamente
\banca{}{}{}{}{}{}{}{}

\defesa{04 de outubro de 2000}

\notaindicativa{Monografia apresentada para obtenção do Grau de
Bacharel em Ciência da Computação pela Universidade Federal do
Paraná.}

\begin{document}

\makecapa
\makerosto         % cria folha de rosto para versao final da UFPR %
%\maketermo      % cria folha com o termo de aprovacao da dissertacao%

%\singlespacing           % espacamento 1 - capa UFPR%
%\onehalfspacing          % espacamento 1/2 %
\doublespacing            % espacamento 2 - UFPR %

\pagestyle{headings}
\pagenumbering{roman}

%\chapter*{Agradecimentos}
%\input{agradecimentos.tex}          % possiu somente o texto

\tableofcontents

%\listoffigures         % se houver mais do que 3 figuras
%\addcontentsline{toc}{chapter}{\MakeUppercase{Lista de Figuras}}
%\newpage

%\listoftables        % se houver mais do que 3 tabelas
%\addcontentsline{toc}{chapter}{\MakeUppercase{Lista de Tabelas}}
%\newpage

\chapter*{Resumo}
\addcontentsline{toc}{chapter}{\MakeUppercase{Resumo}}
%This paper presents a summary, which aims to cover the main ideas available on the
%literature of the past few years regarding network intrusion detection. The goal is to
%explain the general architecture of Network Intrusion Detection Systems (NIDS) and how
%different machine learning algorithms are applied. The focus of this review will be the
%anomaly-based systems, which present a vast field of study.
%
%
%\noindent \textbf{key-words}: nids, network security, intrusion detection, machine learning, data mining, pattern recognition

Este trabalho de graduação tem como intuito apresentar um estudo acerca da usabilidade e viabilidade de
algoritmos de \textit{data mining} e \textit{machine learning} no contexto de sistemas de detecção de
intrusão em rede. Sua utilização teria grande impacto na segurança tanto de servidores quanto de clientes,
pois promete evitar ataques ao sistema detectando anomalias na rede.
Neste artigo são apresentadas técnicas de \textit{machine learning}; a arquitetura geral
de um sistema baseado em anomalia; o processo de extração e processamento de dados; e, por fim, demonstra,
os impactos de uma base de dados representativa, de tipos de ataques desconhecidos e da seleção de atributos.
Os estudos sugerem que a imprevisibilidade de novos tipos de ataques ainda é um empecílio, mesmo para
sistemas com capacidade de aprender ao ser alimentado com novos dados.

\noindent \textbf{Palavras-chave}: NIDS, network security, intrusion detection, machine learning, data mining,
pattern recognition, segurança, aprendizado de máquina, mineiração de dados
           % somente o texto
\newpage


\pagenumbering{arabic}

\chapter{Introdução}
\label{ch:introducao}

Ao longo da última década, a realidade da comunicação em rede entre computadores pessoais e servidores evoluiu
para uma grande parcela da população, deixando de servir apenas para compartilhamento de websites.
Tornou-se um completo ambiente utilizado por cidadãos para armazenamento de arquivos pessoais na núvem;
por bancos para efetuar transações monetárias; e por governos para trocas de informações preciosas.
Assim como é possível se conectar de uma máquina pessoal a um servidor \textit{web}, é possível fazer o contrário.
Qualquer máquina conectada a uma rede está teoricamente sujeita ao acesso de fora.
Roteadores e sistemas operacionais atuais estão equipados com ferramentas básicas para previnir acesso não autorizado
ao computador. Ainda assim, há pessoas dedicadas a burlar tais proteções. Alguns desses invasores conseguem copiar
arquivos, instalar \textit{malware} e até ganhar controle sobre o sistema sem serem notados pelo dono.
\par A fim de cobrir esse ponto cego, foram criados sistemas de detecção de intrusão, ou IDS
(\textit{intrusion detection systems}). Seu objetivo é reconhecer comportamentos incomuns dentro de uma rede.
No meio acadêmico, muitos artigos foram publicados explorando algoritmos de Machine Learning nesta aplicação.
\par Existem diversos tipos de ataques e diferentes protocolos de comunicação, mas o problema de se detectar uma
intrusão pode ser simplificado em um problema de classificação binária do comportamento do sistema, onde os dois
possíveis resultados seriam "normal" ou "intrusivo". Existem duas "escolas de pensamento" bem definidas no meio
acadêmico acerca do assunto:
\begin{enumerate}
    \item \textit{Signature-based} (baseado em assinatura) -- Esse é o método padrão e atualmente o único utilizado
    comercialmente. Ele possui baixa adaptabilidade a novas ameaças.
    \item \textit{Anomaly-based} (baseado em anomalia) -- Ainda uma área de estudo, possui alta taxa de
    Falso Positivo, mas boa adaptabilidade e será o foco deste trabalho.
\end{enumerate}

\chapter{Conclusão}
\label{ch:conclusao}

Em suma, os algoritmos de aprendizado tentam automatizar o processo de reconhecer ameaças dentro da rede
através da detecção de conexões anômalas, dando luz aos Sistemas de Detecção de Intrusão em Rede baseados em Anomalia.
Neste trabalho, foram analisados os diferentes tipos de dados utilizados para a detecção de intrusões em rede e a
performance de algoritmos sugeridos em trabalhos relacionados. Também foram estudados os impactos de uma base de
treino suficientemente representativa da rede para predição de ataques desconhecidos pelo sistema.
\par O primeiro teste, usando apenas SVM, mostrou a necessidade de se gerar uma base de treino grande o suficiente, a
fim de se evitar amostras pouco representativas. Os resultados do segundo teste sugerem que, possivelmente, os A-NIDS
não tenham tanta escalabilidade com o surgimento de ameaças completamente desconhecidas e que, portanto, seriam
necessárias bases específicas para aplicações específicas ao invés de sistemas genéricos. O último teste tinha a
intenção de melhorar os resultados anteriores através da seleção de atributos, mas o seu ganho de performance não foi
significativo.
\par O campo ainda está aberto para novos estudos e análises quanto a viabilidade do uso de machine learning em tempo
real para a detecção de novas ameaças. Existem outros pontos não abordados nesse trabalho que também têm espaço para
serem otimizados, tais quais a atual falta de métricas unificadas; o processamento do alto, rápido e contínuo fluxo de
dados na rede; análise de dados cifrados e a taxa aceitável de de Falso Positivo. Por fim, pode-se considerar que
 o maior desafio nos testes de detecção de intrusão em rede reside em adquirir uma base de treino e de testes que
 represente precisamente o atual comportamento da rede.

% Trocar para ficar no padrão brasileiro
%\bibliographystyle{brazil}
\bibliographystyle{plain}
\bibliography{bib}
% utilize macros (3 primeiras letras do mes em ingles, minusculas) no seu
% .bib para atribuir o nome do mes em portugues nas referencia,
% se o style for brazil, outros estilos tambem aceitam estas macros
% Ex:
%
% @InProceedings{teste,
%   author =       {Luciano}
%   year =         {2000}
%   month =        {}#sep;
% }
%
\addcontentsline{toc}{chapter}{\MakeUppercase{Bibliografia}}

\singlespacing

\end{document}
