\chapter{Anomaly-based IDS}
\label{anids}
 A técnica de detecção de anomalia analisa o atual comportamento da rede para checar se corresponde ou não a um
 comportamento normal. O maior benefício desse método é a habilidade de identificar novos ataques com sucesso.
 A desvantagem é a alta taxa de Falso Positivo. Ao invés de requerer um novo modelo a ser adaptado, sistemas baseados
 em anomalia necessitam de dados de tráfego de rede sem livres de ataques. \textit{Data mining} também tem sido
 utilizado em sistemas baseados em anomalia, junto a outros métodos estatísticos e de \textit{Machine Learning}
 (aprendizado de máquina).
 \par As diferentes técnicas podem ser agrupadas em três categorias, de acordo com o processamento envolvido:
 estatístico, knowledge-based (baseado em conhecimento) ou algoritmos de aprendizado de máquina.

\section{Técinas Estatísticas}
Um modelo de probabilidade de um determinado comportamento é criado a partir da atividade capturada do tráfego de rede,
 tal como taxa de tráfego, número de pacotes e quantidade de endereços de IP distintos.
 \par Ele não requer conehcimento prévio pois está capacitado a aprender o comportamento normal a partir de observações
 sem ataques. Outro benefício das técnicas estatísticas é a possibilidade de identificar, com precisão, atividades
 maliciosas que ocorrem ao longo de períodos de tempo mais extensos. Por outro lado, nem todos os possíveis
 comportamentos conseguem ser modelados e o equilíbrio entre taxas de Falso Positivo e Falso Negativo dependem
 fortemente na configuração correta dos parâmetros.

 \section{Técnicas Baseadas em Conhecimento}
 Também chamados de expert systems, sistemas baseados em conhecimento são implementados criando-se um conjunto de
 regras de classificação para categorizar os dados. O modelo é geralmente criado por um humano experiente no campo
 da aplicação. Esse tipo de sistema não aponta novas atividades inofensivas como sendo maliciosas, garantindo, assim,
 um número reduzido de Falso Positivo. Por essa razão, o conjunto de regras precisa ser específico o suficiente e,
 apesar de ser possível atingir um certo nível de automatização usando uma máquina de estados finitos, requer
 grande conhecimento sobre o comportamento da rede e tempo significativo para ser desenvolvido.

 \section{Técnicas de Machine Learning}