\chapter{Dados}
\label{dados}

\section{Pré-processamento}
Como anteriormente apontado, sistemas baseados em anomalia estão sujeitos a uma alta taxa de Falso Positivo. Apenas
1\% sequer de Falso Positivo resulta em um número absurdo de falsos alertas. Deve-se ter em mente que servidores lidam
com centenas de conexões simultâneas e inúmeros pacotes a cada centésimo de segundo.
\par O pré-processamento de dados é, então, exigido para atingir uma melhor performance na detecção de intrusão.
O pré-processamento converte tráfego de rede em uma série de ocorrências, onde cada uma é representada por um vetor
de características. As informações seguintes acerca dos dados e características foram baseadas nos estudos presentes
em \cite{davis11}.

\subsection{Fluxo}
Os passos mais relevantes para o modelamento de dados de um NIDS são os seguintes:
\begin{enumerate}
    \item Criação da base -- identifica dados categorizados (normal ou anômalo) representativos para treino e teste.
    Categorização de tráfego de rede pode ser uma tarefa difícil e longa, que geralmente envolve um especialista.
    \item Construção de características -- cria características com maior discriminabilidade. Tais características
    podem ser construídas por um humano ou por algoritmos de aprendizado de máquina.
    \item Redução -- também chamado de "seleção de caterísticas", diminui a dimensionalidade da base de dados
    descartando características irrelevantes ou redundantes. Usado para atenuar a "Maldição da Dimensionalidade"
    \footnote{\textit{Curse of Dimensionality}: Ao se usar muitas características, supostamente se obtém baixa precisão.}