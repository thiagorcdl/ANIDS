\chapter{Dados}
\label{dados}
Os passos mais relevantes para o modelamento de dados de um NIDS são os seguintes:
\begin{enumerate}
    \item Criação da base -- identifica dados categorizados (normal ou anômalo) representativos para treino e teste.
    Categorização de tráfego de rede pode ser uma tarefa difícil e longa, que geralmente envolve um especialista.
    \item Construção de características -- cria características com maior discriminabilidade. Tais características
    podem ser construídas por um humano ou por algoritmos de aprendizado de máquina.
    \item Redução -- também chamado de "seleção de caterísticas", diminui a dimensionalidade da base de dados
    descartando características irrelevantes ou redundantes. Usado para atenuar a "Maldição da Dimensionalidade"
    \footnote{\textit{Curse of Dimensionality}: Ao se usar muitas características,
    supostamente se obtém baixa precisão.}.

\section{Origem dos Dados}
A escolha de informações da rede é amplamente afetada pelos requisitos de detecção ao se projetar o sistema.
É possível ter sistemas de detecção específicos, utilizando um conjunto de características limitado.
Para um sistema mais genérico, o ideal seria utilizar detectores separados utilizando conjuntos de características
distintos, um para cada especificidade. Dependendo da origem desses dados, têm-se algumas vantagens e desvantagens.

\subsection{Cabeçalho de pacote}
Características obtidas através de  cabeçalhos de pacotes têm a qualidade de serem rápidos, sem exigir muito
processamento ou memória, e evitarem preocupações legais acerca de análise de dados da rede.
O conjunto de característica mais simples contém características básicas extraídas dos cabeçalhos. Essas
características podem ser usadas para apontar pacotes individuais que são anômalos quando comaprados ao modelo de
treino; ou como um processo de filtragem para que apenas pacotes incomuns sejam usados por algoritmos de detecção
posteriores.
Entretanto, pacotes individuais não podem ser usados para identificar padrões incomuns durante um grande período.
Existem ataques que contém cabeçalhos normais quando analisados individualmente, enquanto sua repetição durante
um certo tempo pode ser considerada anômala. Um exemplo seria o ataque de negação de serviço, popularmente conhecido
como DoS.

\subsection{Corpo de pacote}
Quando ataques são destinados a aplicações, os bytes maliciosos estão dentro do corpo do
pacote e, portanto, as técnicas baseadas em cabeçalho não podem ser usadas. Isso representa um defeito considerável,
principalmente porque diversos ataques da atualidade não são direcionados à rede em si, mas a aplicações conectadas a
ela.
\par NIDS devem utilizar características baseadas em conteúdo, extraídas do corpo dos pacotes, para detectar tais tipos 
de ataques, uma vez que cabeçalhos podem aparentar completamente normais. Análise de conteúdo é computacionalmente 
mais cara do que análsie de cabeçalho porque requer uma inspeção mais profunda do pacote. Ela lida com uma variedade de 
tipo de conteúdo (pdf, jpg, HTML), compressão, e métodos que encobrem dados. Entretanto, o benefício da análise do
corpo é ter acesso a todos os bytes transferidos entre os aparelhos na rede, permitindo a cosntrução de um rico
conjunto de características baseadas em conteúdo para detecção de anomalia.
\par Como análise de conteúdo possui uma alta complexidade, diversos métodos focam em pequenos subconjuntos de
conteúdo, como requisições HTTP ou apenas o JavaScript de um conteúdo baixado. Métodos baseados em anomalia não tentam
comparar assinatoras de malware conhecido, mas podem aplicar heurísticas, como Casamento de Padrões para detectar a
presença de código shell.

\subsection{Cliente-Servidor}
Analyzing client content
is a less researched field, from an anomaly-NIDS per-
spective. The client techniques intended to detect cur-
rent web threats such as drive-by downloads, cross-
site scripting and other malicious JavaScript


\section{Pré-processamento}
Como anteriormente apontado, sistemas baseados em anomalia estão sujeitos a uma alta taxa de Falso Positivo. Apenas
1\% sequer de Falso Positivo resulta em um número absurdo de falsos alertas. Deve-se ter em mente que servidores lidam
com centenas de conexões simultâneas e inúmeros pacotes a cada centésimo de segundo.
\par O pré-processamento de dados é, então, exigido para atingir uma melhor performance na detecção de intrusão.
O pré-processamento converte tráfego de rede em uma série de ocorrências, onde cada uma é representada por um vetor
de características. As informações seguintes acerca dos dados e características foram baseadas nos estudos presentes
em \cite{davis11}.

\subsection{Derivação}
Different types of features used in anomaly-based
NIDS were covered by [3]. Each feature type is de-
rived from many data preprocessing methods includ-
ing organizing packets into flows, analyzing applica-
tion content for fields of interest or parsing individual
network packet headers.


\section{Conjuntos de Características}
\subsection{Conexões múltiplas}
Great part of the NIDS reviewed in [3] use flow or
session network data. Features are then built from
the flows. The most popular is the packet header
method using multiple connection derived (MCD) fea-
tures. These features are generally derived using
mean, standard deviation, and percentage of flows,
covering multiple sessions. Anomaly-based NIDS us-
ing these features are
\subsection{Única conexão}
Single connection derived (SCD) features are used to
detect anomalous behavior within a single session.
These can highlight an unexpected protocol, uncom-
mon data sizes, unusual packet timing, or uncommon
TCP flag sequences. Thereby, SCD features enable
the detection