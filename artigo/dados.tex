\chapter{Construção da Base de Dados}
\label{ch:dados}
Os passos mais relevantes para o modelamento de dados de um A-NIDS são os seguintes:
\begin{enumerate}
    \item Criação da base -- identifica dados categorizados (normal ou anômalo) representativos para treino e teste.
    Categorização de tráfego de rede pode ser uma tarefa difícil e longa, que geralmente envolve um especialista.
    \item Construção de características -- cria características com maior discriminabilidade. Tais características
    podem ser construídas por um humano ou por algoritmos de aprendizado de máquina.
    \item Redução -- também chamado de "seleção de características", diminui a dimensionalidade da base de dados
    descartando características irrelevantes ou redundantes \cite{song13}. Usado para atenuar a "Maldição da
    Dimensionalidade" \footnote{\textit{Curse of Dimensionality}: Ao se usar muitas características,
    supostamente se obtém baixa precisão.}.
\end{enumerate}

%%%%%%%%%%%%%%%%%%%%%%%%%%%%%%%%%%%%%%%%%%%%%%%%%%%%%%%%%%%%%%%%%%%%%%%%%%%%%%%%%%%%%%%%%%

\section{Origem dos Dados}
A escolha de informações da rede é amplamente afetada pelos requisitos de detecção ao se projetar o sistema.
É possível ter sistemas de detecção específicos, utilizando um conjunto de características limitado \cite{guanzhong09}.
Para um sistema mais genérico, o ideal seria utilizar detectores separados utilizando conjuntos de características
distintos, um para cada especificidade. Dependendo da origem desses dados, têm-se algumas vantagens e desvantagens.
Técnicas para análise de conteúdo ainda não estão tão concretizadas quanto as que extraem características de
cabeçalhos, assim como analisar conteúdo do lado do cliente ainda é um campo não tão estudado quanto o lado do servidor
numa perspectiva de A-NIDS. Técnicas do lado do cliente almejam detectar ameaças em aplicações web, como \textit{drive-by
downloads}, \textit{cross-site scripting} e trechos maliciosos de JavaScript.

\subsection{Cabeçalho de pacote}
O conjunto de característica mais simples contém atributos básicos extraídos dos cabeçalhos.
Características obtidas através de  cabeçalhos de pacotes têm a qualidade de serem de rápida extração, sem exigir muito
processamento ou memória, e evitarem preocupações legais acerca de análise de dados da rede.
Essas características podem ser usadas para apontar pacotes individuais que são anômalos quando comparados ao modelo de
treino; ou como um processo de filtragem para que apenas pacotes incomuns sejam usados por algoritmos de detecção
posteriores.
Entretanto, pacotes individuais não podem ser usados para identificar padrões incomuns durante um grande período.
Existem ataques que contêm cabeçalhos normais quando analisados individualmente, enquanto sua repetição durante
um certo tempo pode ser considerada anômala. Um exemplo seria o ataque de negação de serviço, popularmente conhecido
como \textit{DoS} \cite{catania12}.

\subsection{Corpo de pacote}
Quando ataques são destinados a aplicações, os bytes maliciosos estão dentro do corpo do
pacote e, portanto, as técnicas baseadas em cabeçalho não podem ser usadas. Isso representa um defeito considerável,
principalmente porque diversos ataques da atualidade não são direcionados à rede em si, mas a aplicações conectadas a
ela \cite{yaman11}.
\par NIDS devem utilizar características baseadas em conteúdo, extraídas do corpo dos pacotes, para detectar tais tipos 
de ataques, uma vez que alguns cabeçalhos podem aparentar completamente normais. Análise de conteúdo é computacionalmente
mais cara do que análise de cabeçalho porque requer uma inspeção mais profunda do pacote. Ela lida com uma variedade de
tipo de conteúdo (pdf, jpg, HTML), compressão, e métodos que encobrem dados. Entretanto, o benefício da análise do
corpo é ter acesso a todos os bytes transferidos entre os dispositivos na rede, permitindo a construção de um rico
conjunto de características baseadas em conteúdo para detecção de anomalia.
\par Como análise de conteúdo possui uma alta complexidade, diversos métodos focam em pequenos subconjuntos de
conteúdo, como requisições HTTP ou apenas o JavaScript de um conteúdo baixado. Métodos baseados em anomalia não tentam
comparar assinaturas de malware conhecido, mas podem aplicar heurísticas, como Casamento de Padrões para detectar a
presença de código shell.

%%%%%%%%%%%%%%%%%%%%%%%%%%%%%%%%%%%%%%%%%%%%%%%%%%%%%%%%%%%%%%%%%%%%%%%%%%%%%%%%%%%%%%%%%%

\section{Conjuntos de Características}
\subsection{Conexões múltiplas}
Grande parte dos NIDS analisados em \cite{davis11} usam dados de rede referentes a fluxo ou sessão. Características
são construídas a partir do fluxo. O método mais popular é o de cabeçalho de pacotes utilizando características
derivadas de múltiplas conexões (MCD). Esses atributos são geralmente derivados usando média, desvio padrão, e
porcentagem de fluxos, cobrindo múltiplas sessões. NIDS baseados em anomalia que utilizam essas caracterśiticas são
capazes de discernir entre atividade normal da rede e tráfego incomum como \textit{DoS} e \textit{scanning}.
\par Características MCD são geralmente extraídas a partir de conexões dentro de um intervalo de tempo. A maioria das
 Características de MCD são baseadas em volume, como a quantidade de conexões a um endereço de IP e porta
 em um espaço de tempo. Portanto, atributos derivados de múltiplas conexões podem ser facilmente usados para detectar
 volumes excessivos de tráfego relacionados a \textit{DoS} e \textit{probing}. Como pacotes anômalos individuais não
 suprem o valor baseado em volume, eles podem ser ignorados.

\subsection{Conexão individual}
Características derivadas de conexões individuais (SCD) são utilizadas para detectar comportamento anômalo dentro de
uma úncia sessão \cite{song13}. Elas podem apontar um protocolo inesperado, tamanhos incomuns de dados, \emph{timestamp} não condizente,
ou sequências incomuns de \textit{flags} de TCP. Portanto, características SCD permitem detecção de uso anômalo da rede
por \textit{backdoors}, túnel HTTP e afins.
\par As características SCD fornecem contexto que pode ser usado para encontrar anomalias não contextuais. Por exemplo, se a
temporização de pacotes dentro de uma porta monitorada não se encaixa com um perfil esperado, um alerta pode ser
disparado, uma vez que pode se tratar de um protocolo de tunelamento.

\subsection{KDD Cup 99}
O KDD Cup 99 se trata de uma base de dados para uso livre. Ela possui 32 características quantitativas e
9 qualitativas, junto a tráfego normal e ataques de \textit{probing}, negação de serviço, \textit{user to root} e
\textit{remote to local} \cite{zhang09}.  Tais características foram construídas por um especialista e involvem informações de mais
alto nível, como o número de tentativas de login sem sucesso, se acesso root ao terminal foi conseguido,
e o número de operações de criação de arquivo.
\par Considerando que o KDD Cup 99 já tem mais de 15 anos, é provável que essas características não sejam mais úteis
para detectar os tipos de ataques recentes na realidade do fluxo de dados na rede atualmente. Para se detectar
ameaças atuais, é necessário construir novas características baseadas em conteúdo, uma vez que o conteúdo em si mudou.

%%%%%%%%%%%%%%%%%%%%%%%%%%%%%%%%%%%%%%%%%%%%%%%%%%%%%%%%%%%%%%%%%%%%%%%%%%%%%%%%%%%%%%%%%%

\section{Pré-processamento}
Como anteriormente apontado, sistemas baseados em anomalia estão sujeitos a uma alta taxa de Falso Positivo. Apenas
1\% sequer de Falso Positivo resulta em um número absurdo de falsos alertas. Deve-se ter em mente que servidores lidam
com centenas de conexões simultâneas e inúmeros pacotes a cada centésimo de segundo.
\par O pré-processamento de dados é, então, exigido para atingir um melhor desempenho na detecção de intrusão.
O pré-processamento converte tráfego de rede em uma série de ocorrências, onde cada uma é representada por um vetor
de características. As informações seguintes acerca dos dados e características foram baseadas nos estudos presentes
em \cite{davis11}.

\subsection{Derivação}
Diferentes tipos de características que podem ser usados em NIDS baseados em anomalia foram cobertos em \cite{davis11}.
Cada tipo de característica é derivada de vários métodos de pré-processamento, incluindo organizar pacotes em fluxos,
analisar conteúdo de aplicações em busca de campos de interesse ou percorrer cabeçalhos de cada pacote. A derivação do
potencial conjunto de características é um passo essencial para o NIDS baseado em anomalia. Analogamente,
pré-processamentos subsequentes podem ajudar ainda mais a aumentar a eficiência do NIDS.
\par Métodos de pré-processamento de mineração de dados podem ser usados, incluindo transformação, redução e
discretização de dados, como os apresentados em \cite{ribeiro08}. Uma técnica de redução geralmente utilizada é a
\textit{Principal Component Analysis}. PCA tem se mostrado útil para redução da dimensionalidade dos dados,
consequentemente reduzindo o custo computacional do sistema de detecção
\par Para eliminar características redundantes, diversos algoritmos automatizados para seleção de características
também existem, resultando em redução de dados similar. Tais métodos de redução de dados fornecem um meio preciso para
a conversão de um conjunto candidato para o conjunto final de características. É possível que métodos automáticos de
redução possam melhorar ainda mais o NIDS, apesar de que alguns sistemas são construídos apenas se baseando no
conhecimento de um especialista para construir conjuntos de características significativos.

\subsection{Alternativas de processamento}
Considerando métodos que almejam maior automação no momento do processamento dos dados e características, existem
duas ferramentas aparentemente promissoras: N-grams e libAnomaly. N-grams pode ser usado com algoritmos de seleção
automática de características, enquanto o libAnomaly se trata de uma biblioteca livre para modelos de pré-processamento
de dados.
\par No N-grams, perfis são construídos para diferentes tipos de arquivos, calculando as distribuições de frequência
de byte \cite{zhuowei03}. O NIDS compara, então, o tráfego observado com os perfis usando Distância Manhattan. N-gram é geralmente
usado para analisar requisições ao servidor. É útil em detectar padrões anômalos, como código shell dentro de
protocolos estruturados de aplicações, sem exigir muita supervisão.
\par O foco do libAnomaly, por sua vez, é construir um punhado de modelos representando a transferência normal de
dados entre usuário e aplicação web. Requisições maliciosas provavelmente serão apontadas como anômalas quando
comparadas com um ou mais modelos, pois geralmente se diferem de requisições normais de alguma forma.
\par Embora a intenção é que NIDS sejam completamente automáticos, sua criação e adaptação requerem um especialista.
Mesmo o N-grams necessita de auxílio em algumas áreas da rede \cite{zhuowei03}. LibAnomaly também requer experiência significativa
para selecionar quais campos dentro da rede devem ser modelados. Por outro lado, detecção usando cabeçalhos de
pacotes requerem menos conhecimento. Técnicas de mineração de dados podem ser usadas para derivar características de
cabeçalhos MCD para detectar algumas ameaças. O conjunto de características em potencial  poderá, então, ser processado no
estágio de seleção de característica. Essas técnicas automatizadas garantem que as características disponíveis mais discriminantes sejam
escolhidas.
