%This paper presents a summary, which aims to cover the main ideas available on the
%literature of the past few years regarding network intrusion detection. The goal is to
%explain the general architecture of Network Intrusion Detection Systems (NIDS) and how
%different machine learning algorithms are applied. The focus of this review will be the
%anomaly-based systems, which present a vast field of study.
%
%
%\noindent \textbf{key-words}: nids, network security, intrusion detection, machine learning, data mining, pattern recognition

Este trabalho de graduação tem como intuito apresentar um estudo acerca da usabilidade e viabilidade de
algoritmos de \textit{data mining} e \textit{machine learning} no contexto de sistemas de detecção de
intrusão em rede. Sua utilização teria grande impacto na segurança tanto de servidores quanto de clientes,
pois promete evitar ataques ao sistema detectando anomalias na rede.
Nesta monografia são apresentadas técnicas de \textit{machine learning}; a arquitetura geral
de um sistema baseado em anomalia; o processo de extração e processamento de dados; e, por fim, demonstra
os impactos de uma base de dados representativa, de tipos de ataques desconhecidos e da seleção de atributos.
Os estudos sugerem que a imprevisibilidade de novos tipos de ataques ainda é um empecílio, mesmo para
sistemas com capacidade de aprender ao ser alimentado com novos dados.

\noindent \textbf{Palavras-chave}: NIDS, network security, intrusion detection, machine learning, data mining,
pattern recognition, segurança, aprendizado de máquina, mineiração de dados
