O aumento da resolução espacial de imagens é de grande interesse para várias
aplicações. Dentro da biometria isto pode implicar no aumento da performance
dos algoritmos e do relaxamento do grau de cooperação necessário dos indivíduos
a serem identificados. A super-resolução surge como meio de realização desta
tarefa, alternativa à acquisição de novos sensores de custo mais elevado. O
método consiste em combinar os dados contidos em diversas vistas de uma mesma
cena, sejam elas capturadas por um único ou múltiplos sensores, a fim de gerar
um quadro de maior resolução como saída. Sistemas que utilizam informação 3D
também podem ser beneficiados, pois quando aplicado a imagens de profundidade,
a super-resolução cria nuvens de pontos de maior densidade. Neste trabalho foi
avaliada a possibilidade da aplicação de técnicas de super-resolução em
sistemas biométricos 2D e 3D, com ênfase em sistemas de reconhecimento facial.
No caso 2D, um algoritmo capaz de obter resultados melhores que a interpolação
bicúbica foi implementado. Para 3D, uma análise visual nos resultados do nosso
algoritmo aplicado em objetos rígidos mostrou potencial para aplicações
práticas.

\noindent \textbf{Palavras-chave}: Super-Resolução, Imagens de Profundidade,
Biometria, Reconhecimento Facial
