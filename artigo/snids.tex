\chapter{Signature-based NIDS}
\label{ch:snids}
Apesar de não ser o foco desta monografia, vamos descrever brevemente os sistemas baseados em assinatura, a fim de
comparação com os métodos nos quais vamos nos aprofundar.
\par Sistemas baseados em assinatura têm sido os mais bem sucedidos para detecção de intrusão até hoje.
A ideia é comparar o tráfego da rede com padrões de comportamento conhecidos durante certos ataques.
O maior problema é a ineficiência em detectar ataques novos.
\par Os primeiros sistemas utilizavam apenas o método de reconhecimento de padrão. Essa técnica possui um banco de dados
 preenchido com assinaturas de cada ameaça conhecida. Um evento malicioso é detectado se o atual estado da rede se iguala
 a uma assinatura. Como ele tenta comparar todas as assinaturas e a quantidade de dados transmitidos simultaneamente
cresce a cada ano, o custo computacional tronou-se muito alto.
\par Uma segunda técnica é o método de \textit{implication rules}, ou regras de inferência. Ela fornece um conjunto de
regras que descrevem eventos conhecidos que podem inferir o acontecimento de uma intrusão. De qualquer modo, IDS de
assinatura geralmente necessita de um humano capacitado, que cria um novo conjunto de regras toda vez que um novo
tipo de ataque surge, para adquirir modelos de tráfego.
\par Daí surge a motivação para se utilizar técnicas de \textit{data mining} em sistemas de assinatura. Elas proveem um
 modo de aumentar a automação no momento de construção e de ajuste do modelo. Eles podem adaptar os modelos à medida que
 acessam tráfego de rede contendo novas ameaças ou detectando diferentes versões de um ataque conhecido. Ainda assim,
 é impossível identificar comportamentos maliciosos completamente desconhecidos.