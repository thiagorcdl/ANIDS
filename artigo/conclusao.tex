\chapter{Conclusão}
\label{conclusao}

Neste trabalho foram realizados estudos sobre o tema super-resolução de forma a
avaliar sua aplicabilidade em dois cenários. O primeiro em imagens RGB de
sequência de vídeo contendo uma pessoa, de forma a fornecer quadros de maior
resolução a sistemas biométricos. Segundo em imagens de profundidade com
objetos rígidos, a fim de verificar o potencial da aplicação futura com
biometria.

Através da codificação de dois algoritmos, um para cada modalidade,
mostrou-se que o método, mesmo baseado em um modelo simplificado de observação,
permite a obtenção de resultados melhores no aumento da resolução espacial de
imagens quando comparado a soluções tradicionais, como a interpolação bicúbica.

A confirmação da eficácia da super-resolução em sequências de vídeo contendo
faces indica o potencial de sua aplicação em sistemas biométricos. Podendo
tanto diminuir o custo, quanto permitir maior performance.

Em trabalhos futuros espera-se considerar outros tipos de movimento além da
translação; tratar oclusões; experimentar e comparar outros métodos de
estimação de movimento na etapa de alinhamento; avaliar os resultados obtidos
com diferentes métodos de reconstrução; incluir estimação de ruído e desfoque;
e quantificar a influência da super-resolução no aumento da performance de
sistemas biométricos, como por exemplo, de reconhecimento facial.
