\chapter{Conclusão}
\label{ch:conclusao}

Em suma, os algoritmos de aprendizado tentam automatizar o processo de reconhecer ameaças dentro da rede
através da detecção de conexões anômalas, dando luz aos Sistemas de Detecção de Intrusão em Rede baseados em Anomalia.
Neste trabalho, foram analisados os diferentes tipos de dados utilizados para a detecção de intrusões em rede e a
performance de algoritmos sugeridos em trabalhos relacionados. Também foram estudados os impactos de uma base de
treino suficientemente representativa da rede para predição de ataques desconhecidos pelo sistema.
\par O primeiro teste, usando apenas SVM, mostrou a necessidade de se gerar uma base de treino grande o suficiente, a
fim de se evitar amostras pouco representativas. Os resultados do segundo teste sugerem que, possivelmente, os A-NIDS
não tenham tanta escalabilidade com o surgimento de ameaças completamente desconhecidas e que, portanto, seriam
necessárias bases específicas para aplicações particulares ao invés de sistemas genéricos. O último teste tinha a
intenção de melhorar os resultados anteriores através da seleção de atributos, mas o seu ganho de performance não foi
significativo.
\par O campo ainda está aberto para novos estudos e análises quanto a viabilidade do uso de machine learning em tempo
real para a detecção de novas ameaças. Existem outros pontos não abordados nesse trabalho que também têm espaço para
serem otimizados, tais quais a atual falta de métricas unificadas; o processamento do alto, rápido e contínuo fluxo de
dados na rede; análise de dados cifrados e a taxa aceitável de Falso Positivo. Por fim, pode-se considerar que
 o maior desafio nos testes de detecção de intrusão em rede reside em adquirir uma base de treino e de testes que
 represente precisamente o atual comportamento da rede.