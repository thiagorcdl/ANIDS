\chapter{Análise de Trabalhos Relacionados}
\label{analise}

No meio acadêmico, dois artigos recentes trouxeram implementações promissoras a respeito de sistemas de detecção de
anomalia. Um deles (\cite{lin12}) apresenta um algoritmo inteligente e o outro (\cite{papadonikolakis12}) mostra o
ganho de performance ao se programar o sistema diretamente em uma placa FPGA.

\section{Processando grandes quantidades de dados}
A simulação apresentada em \cite{lin12} usou o a base do KDD Cup 99. Ela possui 32 características quantitativas e
9 qualitativas, junto a tráfego normal e ataques de \textit{probing}, negação de serviço, \textit{user to root} e
\textit{remote to local}. O algoritmo funciona pré-processando os dados de treino e teste e gerando uma solução
inicial aleatória. A solução é atualizada a cada iteração do algoritmo. Enquanto um limite não é satisfeito,
são utilizados Support Vector Machine (SVM) e Simulated Annealing (SA) para selecionar o melhor conjunto de
características. Por conseguinte, Árvore de Decisão e SA são usados para aumentar a precisão do teste e construir
regras de decisão. No final, a melhor precisão e as melhores características e regras de decisão são anunciadas.

\subsection{Support Vector Machine}
SVM trabalha mapeando a base de treino em um espaço vetorial de maior dimensionalidade para aumentar a discriminação.
Ele se baseia em minimizar o risco estrutural

\subsection{Árvore de Decisão}