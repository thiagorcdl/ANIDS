\documentclass[12pt,a4paper]{article}

\usepackage[brazil]{babel}
\usepackage[utf8]{inputenc}
\usepackage{multirow}
\usepackage{subcaption}
\usepackage{caption}
\usepackage[table,xcdraw]{xcolor}

\begin{document}

\section{Apresentação}
    Este trabalho de graduação tem como intuito apresentar um estudo acerca da usabilidade e viabilidade de
    algoritmos de \textit{machine learning} em sistemas de detecção de intrusão em rede.
    Sua utilização teria grande impacto na segurança tanto de servidores quanto de clientes,
    pois promete evitar ataques ao sistema detectando anomalias na rede.

    Vamos abordar:
    \begin{itemize}
        \item conceitos de sistema de detecção de intrusão e de \textit{machine learning}
        \item a arquitetura geral de um sistema baseado em anomalia
        \item o processo de extração e processamento de dados;
        \item e três testes efetuados, referentes a
        \begin{enumerate}
            \item impacto de uma base de dados representativa
            \item impacto de tipos de ataques desconhecidos
            \item impacto da seleção de atributos.
        \end{enumerate}
        \item conclusão
    \end{itemize}


\section{Intro}
    \subsection{Motivação}
    Pode ser que eu venha a me referir pela sigla NIDS, que significa \textit{Network Intrusion Detection System},
    mas no geral, apenas Sistema de Detecção ou até Sistema, mesmo. \par
    Bom, roteadores e sistemas operacionais atuais vêm equipados com ferramentas básicas para previnir acesso não autorizado
    ao computador. Ainda assim, há pessoas dedicadas a burlar tais proteções. Uma máquina conectada à rede nunca está 100\% segura.
    Recentemente veio à tona o problema do Shellshock, que com certeza já era conhecido em alguns meios.
    Uma coisa é um usuário baixar um vírus no windows que fica abrindo janelas, mas e o quando alguém acessa seu servidor,
    ganha permissão root, e você nem fica sabendo. \par

    \subsection{NIDS}
    A ideia dos sistemas de detecção de intrusão é cobrir esse ponto cego, detectando automaticamente comportamentos
    nocivos na rede, analisando o tráfego e pacotes. Atualmente, os sistemas em uso comercial são os chamados
    baseados em assinatura. Eles tentam casar padrões com um banco de dados de assinaturas de ataques conhecidos.
    O problema é que algumas alterações podem gerar assinaturas diferentes. Daí surgem os baseados em anomalia.


\section{A-NIDS}
    \subsection{Machine Learning}
    A ideia é usar algoritmos de machine learning para que o sistema aprenda o que é o comportamento normal da rede e bloqueie quaisquer
    conexões que sejam divergentes. Machine Learning tem aplicação em diversar áreas, como processamento de imagens e robótica.
    A ideia geral é o seu sistema aprender algo e usar isso para executar tarefas futuras.
    No nosso caso, é o de classificar as conexões como sendo ou "\textbf{normal}" ou "\textbf{anômala}".

    Enquanto alguns detalhes de implementação podem variar, o padrão da esquematização básica de um A-NIDS
    é o seguinte:
    \begin{itemize}
        \item Aquisição de dados de tráfego -- coleta informação sobre os quadros da rede para processamento futuro.
        \item Gerador de características do tráfego -- extrai características do tráfego capturado.
        \item Gerador de modelo de tráfego (Treino) -- contém informação base usada para guiar o detector de incidente.
        \item Detector de incidente (Classificação) -- identifica atividades intrusivas.
        \item Gerenciamento de resposta -- inicia manobras para superar uma intrusão em potencial. É inicializado pelo
        detector de incidente.
    \end{itemize}

    \subsection{Dados}
    Dados extraídos têm grande influência no desempenho do sistema. Nessa área, podemos ter dados extraídos do cabeçalho
    de pacotes ou do corpo de pacotes. E as características derivadas podem ser tanto de uma única conexão, como
    geradas a partir de um conjunto de conexões. \\
    Dados extraídos de cabeçalho são rápidos de serem obtidos e contém informações de origem, destino, portas, e flags
    TCP. Só que se é um ataque destinado a uma aplicação específica, devemos analisar o conteúdo dos pacotes.
    Isso já é bem mais custoso, pois lida com compressão de dados. \\
    Atributos derivados de uma única conexão podem apontar um protocolo inesperado, tamanhos incomuns de dados,
    \emph{timestamp} não condizente. Portanto, características SCD permitem detecção de uso anômalo da rede
    por \textit{backdoors}, túnel HTTP e afins.\\
    Já os derivados de múltiplas conexões são extraídos a partir de conexões dentro de um intervalo de tempo.
    Esses atributos são geralmente derivados usando média, desvio padrão, e porcentagem de fluxo de pacotes.
    Podem ser usados para detectar volumes excessivos de tráfego relacionados a \textit{DoS} e \textit{probing}.


    \subsection{Algoritmos}
    A característica principal dessa técnica é a habilidade de adaptar suas regras de classificação à medida que
    novos dados são recebidos. Isso garante pouca necessidade por intervenção humana, uma vez funcionando. O lado ruim
    do aprendizado de máquina é o alto custo computacional. Os algoritmos mais populares usados em A-NIDS são:
    \begin{itemize}
        \item Rede Bayesiana -- cria uma rede de relações de probabilidade entre características.
        \item Modelos de Markov -- compara a probabilidade dos dados observados com um \emph{threshold} definido.
        \item Redes neurais -- cria uma rede de \emph{perceptrons} com habilidade de se adaptar.
        \item Support Vector Machines -- encontra um hiperplano que eficientemente separa ambas as opções de classificação.
        \item Árvores de Decisão -- cria uma árvore onde cada folha representa uma classe e arestas correspondem a
            diferentes valores de atributos.
    \end{itemize}

\section{Testes}
    \subsection{Representabilidade}
    O primeiro teste consistiu de avaliar a performance da detecção mudando a relação entre a base de treino e de testes.
    Foi usada uma base de dados livre NSL-KDD, com 1.140.000 instâncias. Ela foi dividida em 5 particionamentos, variando
    a porcentagem para treino e a porcentagem para teste. O algoritmo usado foi o SVM.
    Analisando a tabela, podemos ver uma correlação entre o tamanho da base de treino com a porcentagem de instâncias
    categorizadas corretamente. Com metade do tráfego sendo usado para treino, atingiu 99.64\% de acerto.
    Isso pode parecer muito bom, mas em um mundo real, não há como usar 50\% de todas as conexões para treinar o sistema.
    Além disso, o KDD já tem 16 anos, e de lá pra cá, a realidade da comunicação mudou.

    \subsection{Imprevisibilidade}
    Aí vem a teste a real capacidade dos algoritmos de machine learning se adaptarem a novos ataques. Para testar o
    impacto da imprevisibilidade, foi usada uma base gerada por mestrando da PUC, Eduardo Viegas. Através de simulações de
    rede, ele gerou três grupos de dados, contendo conexões normais e anômalas:
    \begin{enumerate}
        \item Inicial - 500.000 instâncias de conexões usando protocolo HTTP e SMTP.
        \item Análogo - 148.343 instâncias contendo ataques parecidos ou derivados dos anteriores.
        \item Desconhecido - 276.548 instâncias com novos ataques e novos protocolos, como SNMP e SSH.
    \end{enumerate}

    O processo consistiu de treinar algoritmos de SVM e DT usando metade da base Inicial e avaliar o desempenho
     ao se classificar cada um dos grupos restantes. As taxas de acerto evidenciam uma grande perda de confiabilidade
     conforme o surgimento de novos ataques, sendo a classificação da base \textit{Desconhecido} praticamente tão
     eficiente quanto uma adivinhação aleatória.

    \subsection{Redução}
    Tendo em vista o alto custo computacional para processar as 51 características iniciais, foi ponderada a viabilidade
    de se utilizar uma seleção de atributos prévia para agilizar o processamento na tomada de decisões. Utilizando o
    PCA, a quantidade de características foi reduzida para 39.
    o resultado dos algoritmos nesse cenário teve nenhum ganho significativo. Enquanto os atributos selecionados podem
    ser mais descritivos para os ataques vistos na base \textit{Inicial}, talvez não sejam interessantes para
    identificar os ataques que residem além daquele horizonte.


\section{Conclusão}
    \par O primeiro teste, usando apenas SVM, mostrou a necessidade de se gerar uma base de treino grande o suficiente, a
    fim de se evitar amostras pouco representativas. Os resultados do segundo teste sugerem que, possivelmente, os A-NIDS
    não tenham tanta escalabilidade com o surgimento de ameaças completamente desconhecidas e que, portanto, seriam
    necessárias bases específicas para aplicações particulares ao invés de sistemas genéricos. O último teste tinha a
    intenção de melhorar os resultados anteriores através da seleção de atributos, mas o seu ganho de performance não foi
    significativo.
    \par O campo ainda está aberto para novos estudos e análises quanto a viabilidade do uso de machine learning em tempo
    real para a detecção de novas ameaças. Existem outros pontos não abordados nesse trabalho que também têm espaço para
    serem otimizados, tais quais a atual falta de métricas unificadas; o processamento do alto, rápido e contínuo fluxo de
    dados na rede; análise de dados cifrados e a taxa aceitável de Falso Positivo. Por fim, pode-se considerar que
     o maior desafio nos testes de detecção de intrusão em rede reside em adquirir uma base de treino e de testes que
 represente precisamente o atual comportamento da rede.

\end{document}